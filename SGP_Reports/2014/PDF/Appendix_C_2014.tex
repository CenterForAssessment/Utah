\documentclass[12pt]{article}
\usepackage[margin=0.915in]{geometry}

%%%
%%%             BEGIN Damian Preamble
%%%   Preface commands for pdfLaTeX output
%%%

%%
%% Calling relevant packages
%%

\usepackage[%
pdfstartview={FitV}, 
colorlinks=true,
menucolor=DarkGray,
linkcolor=MidnightBlue,
citecolor=MidnightBlue,
urlcolor=OrangeRed]{hyperref}
\usepackage{graphicx}
\DeclareGraphicsExtensions{.pdf}

%% From pandoc default: Begin
\usepackage{graphicx}
\makeatletter
\def\maxwidth{\ifdim\Gin@nat@width>\linewidth\linewidth\else\Gin@nat@width\fi}
\def\maxheight{\ifdim\Gin@nat@height>\textheight\textheight\else\Gin@nat@height\fi}
\makeatother
% Scale images if necessary, so that they will not overflow the page
% margins by default, and it is still possible to overwrite the defaults
% using explicit options in \includegraphics[width, height, ...]{}
\setkeys{Gin}{width=\maxwidth,height=\maxheight,keepaspectratio}
 %% From pandoc default: Begin

%\usepackage[hyper]{apacite}
\usepackage[svgnames]{xcolor}
\usepackage{rotating,bm,amsmath,amsfonts,amssymb,indentfirst,lscape,fancybox,fancyvrb,listings,pdfpages}
\usepackage[pagestyles]{titlesec}
%\usepackage{ucs}

%%
%% Format changes for chapter and section commands
%%

%\input{onesidedheader.tex}

%%
%% Chapter declarations
%%


%%
%% Section declarations
%%

%%
%% Subsection declarations
%%


%%
%% Subsubsection declarations
%%

%%
%% Listings setup information
%%

\lstset{language=R, frame=ltrb, framesep=5pt, xleftmargin=12pt, xrightmargin=5pt,
       numbers=none, breaklines=true, fancyvrb=true,
       breakatwhitespace=true, captionpos=b, abovecaptionskip=1.5ex,
       backgroundcolor=\color{Cornsilk},
       basicstyle=\small\color{DarkSlateGrey},
       keywordstyle=\ttfamily\color{DarkSlateGrey},
       identifierstyle=\ttfamily\color{DarkSlateGrey}\bfseries, 
       commentstyle=\color{Fuchsia},
       stringstyle=\ttfamily,
       showstringspaces=false}

%%
%% Header and Footer Specification 
%% TO ACTIVATE THE HEADER/FOOTER, ONE MUST PLACE \pagestyle{plain} 
%% followed by \pagestyle{damian} in the document
%%

\widenhead{0.14in}{0.14in}

\renewpagestyle{plain}{}

\newpagestyle{damian}[\sffamily]{
\headrule
\sethead[\sectiontitle][\chaptertitle][\thepage]
  {\sectiontitle}{\chaptertitle}{\thepage}

\footrule
\setfoot[][\raisebox{-.85ex}[0pt]{\NavigationBar}][]
{}{\raisebox{-.85ex}[0pt]{\NavigationBar}}{}
	\newcommand{\NavigationBar}{%
	  \Acrobatmenu{PrevPage}{Previous}\hspace{.5cm}
	  \Acrobatmenu{NextPage}{Next}\hspace{.5cm}
	  \Acrobatmenu{FirstPage}{First}\hspace{.5cm}
	  \Acrobatmenu{LastPage}{Last}\hspace{.5cm}
	  \Acrobatmenu{GoBack}{Back}\hspace{.5cm}
	  \Acrobatmenu{Quit}{Quit}%
}
}

%%
%% Custom specifications, commands, and colors
%%

\definecolor{DarkGray}{cmyk}{0,0,0,.624}
\newcommand{\R}{{\sffamily \textup{R}}}
\newcommand{\MlwiN}{{\sffamily \textup{MlwiN}}}
\newcommand{\Sweave}{{\sffamily \textup{Sweave}}}
\newcommand{\sssty}{\scriptscriptstyle}
\newcommand{\bigdot}{\ensuremath{\hspace{-.3ex}\bm{.}\hspace{-.05ex}}}
\renewcommand{\abstractname}{Abstract}
\renewcommand{\lstlistingname}{\R~Code Example}
\renewcommand{\arraystretch}{1.2}
\setlength{\fboxsep}{4mm}
\setlength{\fboxrule}{0.4pt}

%%
%%		END Damian Preamble
%%

\hypersetup{%
  pdftitle={ Appendix C to the 2014 Utah Growth Model Report },
  pdfauthor={  Damian W. Betebenner  ,  Adam R. VanIwaarden  ,  \emph{National Center for the Improvement \ of Educational Assessment (NCIEA)}  },
  pdfcreator={ Damian W. Betebenner  , Adam R. VanIwaarden  , \emph{National Center for the Improvement \ of Educational Assessment (NCIEA)}  },
  pdfkeywords={},
  bookmarks=true} % pdfproducer={pdfLaTeX}

\usepackage{caption}
\usepackage{float}
\usepackage{longtable}
\usepackage{booktabs}
\usepackage{subcaption}
\usepackage{dcolumn}

\setcounter{secnumdepth}{3}




%\usepackage{pdfdraftcopy}
\newtheorem{proposition}{Proposition}
\newtheorem{theorem}{Theorem}
\newtheorem{definition}{Definition}
\newtheorem{corollary}{Corollary}
\DeclareMathOperator*{\argmin}{arg\,min}

% \usepackage{bbm}
\DeclareMathAlphabet{\mathbbm}{U}{bbm}{m}{n}
\SetMathAlphabet\mathbbm{bold}{U}{bbm}{bx}{n}
\DeclareMathAlphabet{\mathbbmss}{U}{bbmss}{m}{n}
\SetMathAlphabet\mathbbmss{bold}{U}{bbmss}{bx}{n}
\DeclareMathAlphabet{\mathbbmtt}{U}{bbmtt}{m}{n}


\newcommand{\pl}[1]{\textsf{PL#1}}
\newcommand{\Cov}{\ensuremath{\mbox{\textsf{Cov}}}}
\newcommand{\Diag}{\ensuremath{\mbox{\textsf{Diag}}}}
\newcommand{\Bias}{\ensuremath{\mbox{\textsf{Bias}}}}
\newcommand{\Astar}[1]{\ensuremath{#1^{^*}}}
\thispagestyle{plain}
\pagestyle{damian}

\begin{document}

\title{\textsf{\LARGE Appendix C to the 2014 Utah Growth Model Report  \\\medskip Transition to the SAGE Assessment Program. }}
\author{  Damian W. Betebenner    \\   Adam R. VanIwaarden    \\   \emph{National Center for the Improvement \ of Educational Assessment (NCIEA)}   }

 \date{June 2015} 

\maketitle

\newpage


\section{Introduction}\label{introduction}

In the 2013-2014 academic year, Utah transitioned from its previous Utah
Criterion Referenced Tests (CRT) to the
\href{http://www.schools.utah.gov/assessment/SAGE.aspx}{Student
Assessment of Growth and Excellence (SAGE)}. The transition included
numerous changes to the assessment system including the incorporation of
new performance standards and moving to a vertical scale. The Utah SOE's
goal was to maintain SGP analyses across this transition. Based upon
research and recommendations from the NCIEA, the calculation of SGPs
across this transition was not likely to be problematic:

\begin{enumerate}
\def\labelenumi{\arabic{enumi}.}
\itemsep1pt\parskip0pt\parsep0pt
\item
  The current and previous assessments did not demonstrate pronounced
  floor or ceilings with large (\textgreater{}5\%) of students obtaining
  the LOSS/HOSS.
\item
  The current and previous assessments would likely show similar
  correlations between current and prior scores seen on previous
  assessments.
\end{enumerate}

Following SGP analysis of SAGE data using CRT data as priors, schools
reported anomalies and this led to a larger investigation of results.
The anomalies centered around deviations from previous results leading
to questions about whether the SGPs calculated across the transition
were valid.

Utah SOE asked the NCIEA to investigate further the validity of the SGP
calculations. Validation began by looking at the two items mentioned
above, and also included an investigation of the annual relationship of
school-level median SGPs.

\pagebreak

\section{Ceiling/Floor Effects}\label{ceilingfloor-effects}

The goodness of fit plots provided in the 2014 analyses confirm that no
ceiling/floor effects were present that might distort the SGP analyses.
All fit plots for the 2014 SGP analyses can be found in Appendix A of
the 2014 Utah Growth Model Report. As an example, Figure C.1 shows the
results for 8\(^{th}\) grade ELA. Evidence of ceiling and floor effects
would appear as inordinate numbers of high or low SGPs at the extremes
of the prior scale scores. However, all analyses demonstrate a uniform
distribution of SGPs as expected.

\begin{figure}[htbp]
\centering
\includegraphics{../img/Goodness_of_Fit/ELA.2014/gofSGP_Grade_8.png}
\caption{Goodness of Fit Plot for 2014 8\(^{th}\) Grade ELA.}
\end{figure}

\pagebreak

\section{Test Score Correlations}\label{test-score-correlations}

Correlations between student level current and prior achievement scale
scores were all high and similar to correlations seen previously between
CRT administration. The correlations between these dependent and
independent variables used in the regression analyses for 2011 - 2014
are shown in Tables 1 and 2, and are highly similar before and after the
assessment transition.

\begin{table}[H]
\caption*{\textbf{Table 1:} Grade-Level CRT and SAGE Correlations\label{table1}} 
\begin{center}
\begin{tabular}{llcllll}
\hline\hline
\multicolumn{2}{c}{\bfseries }&\multicolumn{1}{c}{\bfseries }&\multicolumn{4}{c}{\bfseries Academic Year}\tabularnewline
\cline{1-7}
\multicolumn{1}{c}{Content Area}&\multicolumn{1}{c}{Grade}&\multicolumn{1}{c}{}&\multicolumn{1}{c}{2011}&\multicolumn{1}{c}{2012}&\multicolumn{1}{c}{2013}&\multicolumn{1}{c}{2014}\tabularnewline
\hline
ELA& 4&&0.75&0.74&0.74&0.75\tabularnewline
& 5&&0.76&0.76&0.75&0.75\tabularnewline
& 6&&0.77&0.77&0.77&0.76\tabularnewline
& 7&&0.76&0.77&0.76&0.75\tabularnewline
& 8&&0.73&0.77&0.76&0.77\tabularnewline
& 9&&0.72&0.74&0.77&0.76\tabularnewline
&10&&0.72&0.73&0.76&0.75\tabularnewline
&11&&0.75&0.75&0.76&0.75\tabularnewline
Mathematics& 4&&0.77&0.78&0.77&0.79\tabularnewline
& 5&&0.80&0.81&0.80&0.79\tabularnewline
& 6&&0.80&0.80&0.82&0.81\tabularnewline
& 7&&0.71&0.80&0.80&0.80\tabularnewline
Science& 5&&0.76&0.76&0.76&0.74\tabularnewline
& 6&&0.76&0.76&0.78&0.76\tabularnewline
& 7&&0.78&0.76&0.77&0.77\tabularnewline
& 8&&0.80&0.81&0.80&0.80\tabularnewline
\hline
\end{tabular}\end{center}

\end{table}

\begin{table}[H]
\caption*{\textbf{Table 2:} EOCT Subject Correlations Before and After SAGE Transition.\label{table2}} 
\begin{center}
\begin{tabular}{lcllll}
\hline\hline
\multicolumn{1}{c}{\bfseries }&\multicolumn{1}{c}{\bfseries }&\multicolumn{4}{c}{\bfseries Academic Year}\tabularnewline
\cline{1-6}
\multicolumn{1}{c}{Content Area}&\multicolumn{1}{c}{}&\multicolumn{1}{c}{2011}&\multicolumn{1}{c}{2012}&\multicolumn{1}{c}{2013}&\multicolumn{1}{c}{2014}\tabularnewline
\hline
Earth Science&&0.79&0.80&0.79&0.76\tabularnewline
Biology&&0.79&0.80&0.79&0.77\tabularnewline
Chemistry&&0.73&0.73&0.75&0.73\tabularnewline
Physics&&0.70&0.67&0.68&0.68\tabularnewline
\hline
\end{tabular}\end{center}

\end{table}

\section{School-Level Median SGPs}\label{school-level-median-sgps}

Because SGP analyses done in Utah are cohort referenced, the median SGP
by norm group will always be 50. For each school demonstrating a drop
from year to year with respect to their median SGP there will be a
commensurate increase. Thus, anomalies reported with drops would
necessarily be balanced by others going up. This can be see in the box
plots of Figure C.2. Across all the content areas, the difference in
school level median SGP from this year (SAGE) to last (CRT) are centered
around 0 with some schools demonstrating increases and some decreases.
Filtering out schools with medians based upon more than 20 or 50
students (panels (b) and (c) of Figure C.2 respectively) removes schools
with large differences but with the overall pattern remaining of
symmetry around zero in terms of increases/decreases.

\begin{figure}[H]
\caption*{\label{fig:Bidensity} {\bf{Fig. C.2:}} Median SGP by Year and Content Area.}
  \begin{subfigure}[b]{\textwidth}
    \includegraphics[width=\textwidth]{../img/Appendices/Appendix_C/Boxplot_MSGP_by_Year-1.png}
  \end{subfigure}
\end{figure}

\pagebreak

\begin{figure}[H]
  \begin{subfigure}[b]{\textwidth}
    \includegraphics[width=\textwidth]{../img/Appendices/Appendix_C/Boxplot_MSGP_by_Year20-1.png}
  \end{subfigure}
  %
  \begin{subfigure}[b]{\textwidth}
    \includegraphics[width=\textwidth]{../img/Appendices/Appendix_C/Boxplot_MSGP_by_Year50-1.png}
  \end{subfigure}
\end{figure}

The bubblePlots shown in Figures C.3 and C.4 show the 2013 school median
SGP versus the 2014 school median SGP. Bubbles in the upper left/lower
right of the charts correspond to those with larger differences.

\begin{figure}[H]
\caption*{\label{fig:Bidensity} {\bf{Fig. C.3:}} Median SGP by Year and Content Area Bubble Plots for Grade-Level SAGE.}
  \begin{subfigure}[b]{\textwidth}
    \includegraphics[width=\textwidth]{../img/Appendices/Appendix_C/Bubble_Plots/School_2013_2014_ELA_Growth.png}
  \end{subfigure}
\end{figure}

\pagebreak

\begin{figure}[H]
  \begin{subfigure}[b]{\textwidth}
    \includegraphics[width=\textwidth]{../img/Appendices/Appendix_C/Bubble_Plots/School_2013_2014_MATHEMATICS_Growth.png}
  \end{subfigure}
    %
  \begin{subfigure}[b]{\textwidth}
    \includegraphics[width=\textwidth]{../img/Appendices/Appendix_C/Bubble_Plots/School_2013_2014_SCIENCE_Growth.png}
  \end{subfigure}
\end{figure}

\pagebreak

\begin{figure}[H]
\caption*{\label{fig:Bidensity} {\bf{Fig. C.4:}} Median SGP by Year and Content Area Bubble Plots for End-of-Course Tests (EOCT).}
  \begin{subfigure}[b]{\textwidth}
    \includegraphics[width=\textwidth]{../img/Appendices/Appendix_C/Bubble_Plots/School_2013_2014_EARTH_SCIENCE_Growth.png}
  \end{subfigure}
  %
  \begin{subfigure}[b]{\textwidth}
    \includegraphics[width=\textwidth]{../img/Appendices/Appendix_C/Bubble_Plots/School_2013_2014_BIOLOGY_Growth.png}
  \end{subfigure}
\end{figure}

\begin{figure}[H]
  \begin{subfigure}[b]{\textwidth}
    \includegraphics[width=\textwidth]{../img/Appendices/Appendix_C/Bubble_Plots/School_2013_2014_CHEMISTRY_Growth.png}
  \end{subfigure}
  %
  \begin{subfigure}[b]{\textwidth}
    \includegraphics[width=\textwidth]{../img/Appendices/Appendix_C/Bubble_Plots/School_2013_2014_PHYSICS_Growth.png}
  \end{subfigure}
\end{figure}

\pagebreak

To look more closely at specific outliers, Figure C.5 show student level
SGP distribution by grade and content area across years. The box plot
median indicates the median SGP for that grade by content area by year
set of SGPs.

\begin{itemize}
\itemsep1pt\parskip0pt\parsep0pt
\item
  School 1 is a junior high school. For some tests, the median SGP
  increased and others it decreased. For science in particular there was
  a decrease in median SGP, particularly in grade 8 science.
\item
  School 2 is a high school and had mixed results for growth with a drop
  in Biology and an increase in Earth Science and ELA
\item
  School 3 is an elementary school and had uniformly excellent increases
  in growth in 2014.
\item
  School 4 is an elementary school and has been dropping across many
  tests over the last several years with big drops in 2014 in grade 4
  mathematics.
\item
  School 5 is a mixed grade school and has seen mixed results for growth
  with a drop in grade-level sciences and Earth Science.
\end{itemize}

Overall, the results confirm that the SGP analyses are valid and not
invalidated due to the transition from the CRT to SAGE assessments.
However, it is likely that the transition to new assessments and
standards is reflected in the results just discussed with some schools
making the transition much better than others. Next steps, particularly
for schools with big changes, is to investigate exactly what content
areas and grades led to these changes and reflect upon potential root
causes for the change.

\begin{figure}[H]
\caption*{\label{fig:Bidensity} {\bf{Fig. C.5:}} School SGP distribution by content area, year, and grade with student count.}
  \begin{subfigure}[b]{\textwidth}
    \includegraphics[width=\textwidth]{../img/Appendices/Appendix_C/School_MSGP_Dist_Boxplot-1.png}
  \end{subfigure}
\end{figure}

\pagebreak

\begin{figure}[H]
  \begin{subfigure}[b]{\textwidth}
    \includegraphics[width=\textwidth]{../img/Appendices/Appendix_C/School_MSGP_Dist_Boxplot-2.png}
  \end{subfigure}
  %
  \begin{subfigure}[b]{\textwidth}
    \includegraphics[width=\textwidth]{../img/Appendices/Appendix_C/School_MSGP_Dist_Boxplot-3.png}
  \end{subfigure}
\end{figure}

\pagebreak

\begin{figure}[H]
  \begin{subfigure}[b]{\textwidth}
    \includegraphics[width=\textwidth]{../img/Appendices/Appendix_C/School_MSGP_Dist_Boxplot-4.png}
  \end{subfigure}
  %
  \begin{subfigure}[b]{\textwidth}
    \includegraphics[width=\textwidth]{../img/Appendices/Appendix_C/School_MSGP_Dist_Boxplot-5.png}
  \end{subfigure}
\end{figure}

\bibliographystyle{plainnat}
\bibliography{/Library/Frameworks/R.framework/Versions/3.2/Resources/library/SGPreports/rmarkdown/templates/multi_document/resources/educ.bib}




\end{document}


