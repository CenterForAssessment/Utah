\documentclass[12pt]{article}
\usepackage[margin=0.915in]{geometry}

%%%
%%%             BEGIN Damian Preamble
%%%   Preface commands for pdfLaTeX output
%%%

%%
%% Calling relevant packages
%%

\usepackage[%
pdfstartview={FitV}, 
colorlinks=true,
menucolor=DarkGray,
linkcolor=MidnightBlue,
citecolor=MidnightBlue,
urlcolor=OrangeRed]{hyperref}
\usepackage{graphicx}
\DeclareGraphicsExtensions{.pdf}

%% From pandoc default: Begin
\usepackage{graphicx}
\makeatletter
\def\maxwidth{\ifdim\Gin@nat@width>\linewidth\linewidth\else\Gin@nat@width\fi}
\def\maxheight{\ifdim\Gin@nat@height>\textheight\textheight\else\Gin@nat@height\fi}
\makeatother
% Scale images if necessary, so that they will not overflow the page
% margins by default, and it is still possible to overwrite the defaults
% using explicit options in \includegraphics[width, height, ...]{}
\setkeys{Gin}{width=\maxwidth,height=\maxheight,keepaspectratio}
 %% From pandoc default: Begin

%\usepackage[hyper]{apacite}
\usepackage[svgnames]{xcolor}
\usepackage{rotating,bm,amsmath,amsfonts,amssymb,indentfirst,lscape,fancybox,fancyvrb,listings,pdfpages}
\usepackage[pagestyles]{titlesec}
%\usepackage{ucs}

%%
%% Format changes for chapter and section commands
%%

%\input{onesidedheader.tex}

%%
%% Chapter declarations
%%


%%
%% Section declarations
%%

%%
%% Subsection declarations
%%


%%
%% Subsubsection declarations
%%

%%
%% Listings setup information
%%

\lstset{language=R, frame=ltrb, framesep=5pt, xleftmargin=12pt, xrightmargin=5pt,
       numbers=none, breaklines=true, fancyvrb=true,
       breakatwhitespace=true, captionpos=b, abovecaptionskip=1.5ex,
       backgroundcolor=\color{Cornsilk},
       basicstyle=\small\color{DarkSlateGrey},
       keywordstyle=\ttfamily\color{DarkSlateGrey},
       identifierstyle=\ttfamily\color{DarkSlateGrey}\bfseries, 
       commentstyle=\color{Fuchsia},
       stringstyle=\ttfamily,
       showstringspaces=false}

%%
%% Header and Footer Specification 
%% TO ACTIVATE THE HEADER/FOOTER, ONE MUST PLACE \pagestyle{plain} 
%% followed by \pagestyle{damian} in the document
%%

\widenhead{0.14in}{0.14in}

\renewpagestyle{plain}{}

\newpagestyle{damian}[\sffamily]{
\headrule
\sethead[\sectiontitle][\chaptertitle][\thepage]
  {\sectiontitle}{\chaptertitle}{\thepage}

\footrule
\setfoot[][\raisebox{-.85ex}[0pt]{\NavigationBar}][]
{}{\raisebox{-.85ex}[0pt]{\NavigationBar}}{}
	\newcommand{\NavigationBar}{%
	  \Acrobatmenu{PrevPage}{Previous}\hspace{.5cm}
	  \Acrobatmenu{NextPage}{Next}\hspace{.5cm}
	  \Acrobatmenu{FirstPage}{First}\hspace{.5cm}
	  \Acrobatmenu{LastPage}{Last}\hspace{.5cm}
	  \Acrobatmenu{GoBack}{Back}\hspace{.5cm}
	  \Acrobatmenu{Quit}{Quit}%
}
}

%%
%% Custom specifications, commands, and colors
%%

\definecolor{DarkGray}{cmyk}{0,0,0,.624}
\newcommand{\R}{{\sffamily \textup{R}}}
\newcommand{\MlwiN}{{\sffamily \textup{MlwiN}}}
\newcommand{\Sweave}{{\sffamily \textup{Sweave}}}
\newcommand{\sssty}{\scriptscriptstyle}
\newcommand{\bigdot}{\ensuremath{\hspace{-.3ex}\bm{.}\hspace{-.05ex}}}
\renewcommand{\abstractname}{Abstract}
\renewcommand{\lstlistingname}{\R~Code Example}
\renewcommand{\arraystretch}{1.2}
\setlength{\fboxsep}{4mm}
\setlength{\fboxrule}{0.4pt}

%%
%%		END Damian Preamble
%%

\hypersetup{%
  pdftitle={ Appendix B to the 2014 Utah Growth Model Report },
  pdfauthor={  Damian W. Betebenner  ,  Adam R. VanIwaarden  ,  \emph{National Center for the Improvement \ of Educational Assessment (NCIEA)}  },
  pdfcreator={ Damian W. Betebenner  , Adam R. VanIwaarden  , \emph{National Center for the Improvement \ of Educational Assessment (NCIEA)}  },
  pdfkeywords={},
  bookmarks=true} % pdfproducer={pdfLaTeX}

\usepackage{caption}
\usepackage{float}
\usepackage{longtable}
\usepackage{booktabs}
\usepackage{subcaption}
\usepackage{dcolumn}

\setcounter{secnumdepth}{3}




%\usepackage{pdfdraftcopy}
\newtheorem{proposition}{Proposition}
\newtheorem{theorem}{Theorem}
\newtheorem{definition}{Definition}
\newtheorem{corollary}{Corollary}
\DeclareMathOperator*{\argmin}{arg\,min}

% \usepackage{bbm}
\DeclareMathAlphabet{\mathbbm}{U}{bbm}{m}{n}
\SetMathAlphabet\mathbbm{bold}{U}{bbm}{bx}{n}
\DeclareMathAlphabet{\mathbbmss}{U}{bbmss}{m}{n}
\SetMathAlphabet\mathbbmss{bold}{U}{bbmss}{bx}{n}
\DeclareMathAlphabet{\mathbbmtt}{U}{bbmtt}{m}{n}

\begin{document}


\newcommand{\pl}[1]{\textsf{PL#1}}
\newcommand{\Cov}{\ensuremath{\mbox{\textsf{Cov}}}}
\newcommand{\Diag}{\ensuremath{\mbox{\textsf{Diag}}}}
\newcommand{\Bias}{\ensuremath{\mbox{\textsf{Bias}}}}
\newcommand{\Astar}[1]{\ensuremath{#1^{^*}}}
\thispagestyle{plain}
\pagestyle{damian}

\title{\textsf{\LARGE Appendix B to the 2014 Utah Growth Model Report  \\\medskip Norm- and Criterion-Referenced Growth: An Overview of the SGP
Methodology }}
\author{  Damian W. Betebenner    \\   Adam R. VanIwaarden    \\   \emph{National Center for the Improvement \ of Educational Assessment (NCIEA)}   }

 \date{May 2015} 

\maketitle

\newpage


\section{Introduction - Why Student
Growth?}\label{introduction---why-student-growth}

Accountability systems constructed according to federal adequate yearly
progress (AYP) requirements currently rely upon annual ``snap-shots'' of
student achievement to make judgments about school quality. Since their
adoption, such \emph{status measures} have been the focus of persistent
criticism (Linn, 2003; Linn, Baker, \& Betebenner, 2002). Though
appropriate for making judgments about the achievement level of students
at a school for a given year, they are inappropriate for judgments about
educational \emph{effectiveness}. In this regard, status measures are
blind to the possibility of low achieving students attending effective
schools. It is this possibility that has led some critics of No Child
Left Behind (NCLB) to label its accountability provisions as unfair and
misguided and to demand the use of growth analyses as a better means of
auditing school quality.

A fundamental premise associated with using student growth for school
accountability is that ``good'' schools bring about student growth in
excess of that found at ``bad'' schools. Students attending such schools
- commonly referred to as highly effective/ineffective schools - tend to
demonstrate extraordinary growth that is causally attributed to the
school or teachers instructing the students. The inherent believability
of this premise is at the heart of current enthusiasm to incorporate
growth into accountability systems. It is not surprising that the
November 2005 announcement by Secretary of Education Spellings for the
Growth Model Pilot Program (GMPP) permitting states to use growth model
results as a means for compliance with NCLB achievement mandates and the
Race to the top competitive grants program were met with great
enthusiasm by states (Spellings, 2005).

Following these use cases, the primary thrust of growth analyses over
the last decade has been to determine, using sophisticated statistical
techniques, the amount of student progress/growth that can be
justifiably attributed to the school or teacher - that is, to
disentangle current \emph{aggregate} level achievement from
effectiveness (Ballou, Sanders, \& Wright, 2004; Braun, 2005;
Raudenbush, 2004; Rubin, Stuart, \& Zanutto, 2004). Such analyses, often
called \emph{value-added} analyses, attempt to estimate the teacher or
school contribution to student achievement. This contribution, called
the \emph{school} or \emph{teacher effect}, purports to quantify the
impact on achievement that this school or teacher would have, on
average, upon similar students assigned to them for instruction.
Clearly, such analyses lend themselves to accountability systems that
hold schools or teachers responsible for student achievement.

Despite their utility in high stakes accountability decisions, the
causal claims of teacher/school effectiveness addressed by value-added
models (VAM) often fail to address questions of primary interest to
education stakeholders. For example, VAM analyses generally ignore a
fundamental interest of stakeholders regarding student growth: How much
growth did a student make? The disconnect reflects a mismatch between
questions of interest and the statistical model employed to answer those
questions. Along these lines, (Harris, 2007) distinguishes value-added
for program evaluation (VAM-P) and value-added for accountability
(VAM-A) - conceptualizing accountability as a difficult type of program
evaluation. Indeed, the current climate of high-stakes, test-based
accountability has blurred the lines between program evaluation and
accountability. This, combined with the emphasis of value-added models
toward causal claims regarding school and teacher effects has skewed
discussions about growth models toward causal claims at the expense of
description. Research (Yen, 2007) and personal experience suggest
stakeholders are more interested in the reverse: description first that
can be used secondarily as part of causal fact finding.

In a survey conducted by Yen(2007), supported by the author's own
experience working with state departments of education to implement
growth models, parents, teacher, and administrators were asked what
``growth'' questions were most of interest to them.

\begin{itemize}
\itemsep1pt\parskip0pt\parsep0pt
\item
  \textbf{Parent Questions:}

  \begin{itemize}
  \itemsep1pt\parskip0pt\parsep0pt
  \item
    Did my child make a year's worth of progress in a year?
  \item
    Is my child growing appropriately toward meeting state standards?
  \item
    Is my child growing as much in Math as Reading?
  \item
    Did my child grow as much this year as last year?
  \end{itemize}
\item
  \textbf{Teacher Questions:}

  \begin{itemize}
  \itemsep1pt\parskip0pt\parsep0pt
  \item
    Did my students make a year's worth of progress in a year?
  \item
    Did my students grow appropriately toward meeting state standards?
  \item
    How close are my students to becoming Proficient?
  \item
    Are there students with unusually low growth who need special
    attention?
  \end{itemize}
\item
  \textbf{Administrator Questions:}

  \begin{itemize}
  \itemsep1pt\parskip0pt\parsep0pt
  \item
    Did the students in our district/school make a year's worth of
    progress in all content areas?
  \item
    Are our students growing appropriately toward meeting state
    standards?
  \item
    Does this school/program show as much growth as that one?
  \item
    Can I measure student growth even for students who do not change
    proficiency categories?
  \item
    Can I pool together results from different grades to draw summary
    conclusions?
  \end{itemize}
\end{itemize}

As Yen remarks, all these questions rest upon a desire to understand
whether observed student progress is ``reasonable or appropriate'' (Yen,
2007). More broadly, the questions seek a description rather than a
parsing of responsibility for student growth. Ultimately, questions may
turn to who/what is responsible. However, as indicated by this list of
questions, they are not the starting point for most stakeholders.

In the following paragraphs, student growth percentiles and percentile
growth projections/trajectories are introduced as a means of
understanding student growth in both norm-referenced and criterion
referenced ways. With these values calculated we show how growth data
can be utilized in both a norm- and in a criterion-referenced manner to
inform discussion about education quality. We assert that the
establishment of a norm-referenced basis for student growth eliminates a
number of the problems of incorporating growth into accountability
systems providing needed insight to various stakeholders by addressing
the basic question of how much a student has progressed (Betebenner,
2008; D. W. Betebenner, 2009).

\pagebreak

\section{Student Growth Percentiles}\label{student-growth-percentiles}

It is a common misconception that to quantify student progress in
education, the subject matter and grades over which growth is examined
must be on the same scale - referred to as a vertical scale. Not only is
a vertical scale not necessary, but its existence obscures concepts
necessary to fully understand student growth. Growth, fundamentally,
requires change to be examined for a single construct like math
achievement across time - \emph{growth in what?}

Consider the familiar situation from pediatrics where the interest is on
measuring the height and weight of children over time. The scales on
which height and weight are measured possess properties that educational
assessment scales aspire towards but can never meet.\footnote{The scales
  on which students are measured are often assumed to possess properties
  similar to height and weight but they don't. Specifically, scales are
  assumed to be interval where it is assumed that a difference of 100
  points at the lower end of the scale refers to the same difference in
  ability/achievement as 100 points at the upper end of the scale. (See
  Lord, 1975; and Yen, 1986 for more detail on the interval scaling in
  educational measurement.)}

\begin{quote}
An infant male toddler is measured at 2 and 3 years of age and is shown
to have grown 4 inches. The magnitude of increase - 4 inches - is a well
understood quantity that any parent can grasp and measure at home using
a simple yardstick. However, parents leaving their pediatrician's office
knowing only how much their child has grown would likely be wanting for
more information. In this situation, parents are not interested in an
absolute criterion of growth, but instead in a norm-referenced criterion
locating that 4 inch increase alongside the height increases of similar
children. Examining this height increase relative to the increases of
similar children permits one to diagnose how (a)typical such an increase
is.
\end{quote}

Given this reality in the examination of change where scales of
measurement are perfect, we argue that it is unreasonable to think that
in education, where scales are at best quasi-interval (Lord, 1975; Yen,
1986) one can/should examine growth differently.

Going further, suppose that scales did exist in education similar to
height/weight scales that permitted the calculation of absolute measures
of annual academic growth for students. The response to a parent's
question such as, ``How much did my child progress?'', would be a number
of scale score points - an answer that would leave most parents confused
wondering whether the number of points is good or bad. As in pediatrics,
the search for a description regarding changes in achievement over time
(i.e., growth) is best served by considering a norm-referenced
quantification of student growth - \emph{a student growth percentile}
(Betebenner, 2008; D. W. Betebenner, 2009).

A student's growth percentile (SGP) describes how (a)typical a student's
growth is by examining his/her current achievement relative to his/her
\emph{academic peers} - those students beginning at the same place. That
is, a student growth percentile examines the current achievement of a
student relative to other students who have, in the past, ``walked the
same achievement path''. Heuristically, if the state assessment data set
were extremely large (in fact, infinite) in size, one could open the
infinite data set and select out those students with the exact same
prior scores and compare how the selected student's current year score
compares to the current year scores of those students with the same
prior year's scores - his/her academic peers. If the student's current
year score exceeded the scores of most of his/her academic peers, in a
norm-referenced sense they have done as well. If the student's current
year score was less than the scores of his/her academic peers, in a
norm-referenced sense they have not done as well.

The four panels of Figure B.1. depict what a student growth percentile
represents in a situation considering students having only two
consecutive achievement test scores.

\begin{itemize}
\itemsep1pt\parskip0pt\parsep0pt
\item
  \textbf{Upper Left Panel} Considering all pairs of 2011 and 2012
  scores for all students in the state yields a bivariate (two variable)
  distribution. The higher the distribution, the more frequent the pair
  of scores.\\
\item
  \textbf{Upper Right Panel} Taking account of prior achievement (i.e.,
  conditioning upon prior achievement) fixes the value of the 2011 scale
  score (in this case at approximately 460) and is represented by the
  red slice taken out of the bivariate distribution.
\item
  \textbf{Lower Left Panel} Conditioning upon prior achievement defines
  a \emph{conditional distribution} which represents the distribution of
  outcomes on the 2012 test assuming a 2011 score of 460. This
  distribution is indicated by the solid red slice of the
  distribution.\\
\item
  \textbf{Lower Right Panel} The conditional distribution provides the
  context against which a student's 2012 achievement can be examined and
  provides the basis for a norm-referenced comparison. Students with
  achievement in the upper tail of the conditional distribution have
  demonstrated high rates of growth relative to their academic peers
  whereas those students with achievement in the lower tail of the
  distribution have demonstrated low rates of growth. Students with
  current achievement in the middle of the distribution could be
  described as demonstrating ``average'' or ``typical'' growth. In the
  figure provided the student scores approximately 500 on the 2012 test.
  Within the conditional distribution, the value of 500 lies at the
  75\(^{th}\) percentile. Thus the student's progress from 460 in 2011
  to 500 in 2012 met or exceeded that of 75 percent of students starting
  from the same place. It is important to note that qualifying a student
  growth percentile as ``adequate'', ``good'', or ``enough'' is a
  standard setting procedure that requires stakeholders to examine a
  student's growth \emph{vis-a-vis} external criteria such as
  performance standards/levels.
\end{itemize}

\pagebreak

\begin{figure}[H]
\caption*{\label{fig:Bidensity} {\bf{Fig. B.1:}} Depiction of the distribution associated with 2011 and 2012 student scale scores together with the conditional distribution and associated growth percentile.}
  \begin{subfigure}[b]{0.5\textwidth}
    \includegraphics[width=\textwidth]{../img/Appendices/SGP_Method/bidensity_p1.jpg}
  \end{subfigure}
  %
  \begin{subfigure}[b]{0.5\textwidth}
    \includegraphics[width=\textwidth]{../img/Appendices/SGP_Method/bidensity_p2.jpg}
  \end{subfigure}
  \begin{subfigure}[b]{0.5\textwidth}
    \includegraphics[width=\textwidth]{../img/Appendices/SGP_Method/bidensity_p3.jpg}
  \end{subfigure}
  %
  \begin{subfigure}[b]{0.5\textwidth}
    \includegraphics[width=\textwidth]{../img/Appendices/SGP_Method/bidensity_p4.jpg}
  \end{subfigure}
\end{figure}

Figure B.1 also serves to illustrate the relationship between the
state's assessment scale and student growth percentiles. The scale
depicted in the panels of Figure B.1 is not vertical. Thus the
comparisons or subtraction of scale scores for individual students is
not supported. However, were such a scale in place, the figure would not
change. With or without a vertical scale, the conditional distribution
can be constructed.

In situations where a vertical scale exists, the increase/decrease in
scale score points can be calculated and the growth percentile can be
understood alongside this change. For example, were the scales presented
in Figure B.1 vertical, then one can calculate that the student grew 40
points (from 460 to 500) between 2011 and 2012. This 40 points
represents the absolute magnitude of change. Quantifying the magnitude
of change is scale dependent. For example, different vertical
achievement scales in 2011 and 2012 would yield different annual scale
score increases: A scale score increase of 40 could be changed to a
scale score increase of 10 using a simple transformation of the vertical
scale on which all the students are measured. However, relative to other
students, their growth has not changed - their growth percentile is
invariant to scale transformations common in educational assessment.
Student growth percentiles norm-referencedly situate achievement change
bypassing questions associated with the magnitude of change, and
directing attention toward relative standing which, we would assert, is
what stakeholders are most interested in.

To fully understand how many states intend to use growth percentiles to
make determinations about whether a student's growth is sufficient, the
next section details specifics of how student growth percentiles are
calculated. These calculations are subsequently used to calculate
percentile growth projections/trajectories that are used to establish
how much growth it will take for each student to reach his/her
achievement targets.

\pagebreak

\section{SGP Calculation}\label{sgp-calculation}

Quantile regression is used to establish curvilinear functional
relationships between the cohort's prior scores and their current
scores. Specifically, for each grade by subject cohort, quantile
regression is used to establish 100 (1 for each percentile) curvilinear
functional relationships between the students prior score(s) and their
current score. For example, consider 7\(^{th}\) graders in 2014. Their
grade 3, grade 4, grade 5, and grade 6 prior scores are used to describe
the current year grade 7 score distribution.\footnote{For the
  mathematical details underlying the use of quantile regression in
  calculating student growth percentiles, see the \emph{SGP Estimation}
  section} The result of these 100 separate analyses is a single
coefficient matrix that can be employed as a look-up table relating
prior student achievement to current achievement for each percentile.
Using the coefficient matrix, one can plug in \emph{any} grade 3, 4, 5,
and 6 prior score combination to the functional relationship to get the
percentile cutpoints for grade 7 conditional achievement distribution
associated with that prior score combination. These cutpoints are the
percentiles of the conditional distribution associated with the
individual's prior achievement. Consider a student with the following
mathematics scores:

\begin{table}[H]
\caption*{\textbf{Table 1:} Scale scores for a hypothetical student across 5 years in mathematics.\label{table1}} 
\begin{center}
\begin{tabular}{rrrrr}
\hline\hline
\multicolumn{1}{c}{Grade 3/2010}&\multicolumn{1}{c}{Grade 4/2011}&\multicolumn{1}{c}{Grade 5/2012}&\multicolumn{1}{c}{Grade 6/2013}&\multicolumn{1}{c}{Grade 7/2014}\tabularnewline
\hline
$819$&$818$&$822$&$834$&$836$\tabularnewline
\hline
\end{tabular}\end{center}

\end{table}

Using the coefficient matrix derived from the quantile regression
analyses based upon grade 3, 4, 5, and 6 scale scores as independent
variables and the grade 7 scale score as the dependent variable together
with this student's vector of grade 3, 4, 5, and 6 grade scale scores
provides the scale score percentile cutpoints associated with the grade
7 conditional distribution for these prior scores.

\begin{table}[H]
\caption*{\textbf{Table 2:} Percentile cutscores for grade 7 mathematics based upon the grade 3, 4, 5, and 6 mathematics scale scores given in Table 1.\label{table2}} 
\begin{center}
\begin{tabular}{llllllllllllllll}
\hline\hline
\multicolumn{1}{c}{1st}&\multicolumn{1}{c}{2nd}&\multicolumn{1}{c}{3rd}&\multicolumn{1}{c}{...}&\multicolumn{1}{c}{10th}&\multicolumn{1}{c}{...}&\multicolumn{1}{c}{25th}&\multicolumn{1}{c}{...}&\multicolumn{1}{c}{50th}&\multicolumn{1}{c}{51th}&\multicolumn{1}{c}{...}&\multicolumn{1}{c}{75th}&\multicolumn{1}{c}{...}&\multicolumn{1}{c}{90th}&\multicolumn{1}{c}{...}&\multicolumn{1}{c}{99th}\tabularnewline
\hline
804.8&814.9&819.9&...&825.9&...&830.8&...&835.5&836.3&...&868.9&...&887.1&...&909.8\tabularnewline
\hline
\end{tabular}\end{center}

\end{table}

The percentile cutscores for 7\(^{th}\) grade mathematics in Table FALSE
are used with the student's \emph{actual} grade 7 mathematics scale
score to establish his/her growth percentile. In this case, the
student's grade 7 scale score of 836 lies above the 50\(^{th}\)
percentile cut and below the 51\(^{st}\) percentile cut, yielding a
growth percentile of 50. Thus, the progress demonstrated by this student
between grade 6 and grade 7 exceeded that of 50 percent of his/her
academic peers - those students with the same achievement history.
States can qualify student growth by defining ranges of growth
percentiles. For example, the Utah Growth Model designates growth
percentiles between 35 and 65 as being \emph{typical}. Using Table
FALSE, another student with the exact same grade 3, 4, 5, and 6 prior
scores but with a grade 7 scale score of 804, would have a growth
percentile of 1, which is designated as \emph{low}.

This example provides the basis for beginning to understand how growth
percentiles in the SGP Methodology are used to determine whether a
student's growth is \emph{(in)adequate}. Suppose that in grade 6 a
one-year (i.e., 7\(^{th}\) grade) achievement goal/target of proficiency
was established for the student. Using the lowest proficient scale score
for 7\(^{th}\) grade mathematics, this target corresponds to a scale
score of 900. Based upon the results of the growth percentile analysis,
this one year target corresponds to 95\(^{th}\) percentile growth. Their
growth, obviously, is less than this and the student has not met this
individualized growth standard.

\pagebreak

\section{SGP Estimation}\label{sgp-estimation}

Calculation of a student's growth percentile is based upon the
estimation of the conditional density associated with a student's score
at time \(t\) using the student's prior scores at times
\(1, 2, \ldots, t-1\) as the conditioning variables. Given the
conditional density for the student's score at time \(t\), the student's
growth percentile is defined as the percentile of the score within the
time \(t\) conditional density. By examining a student's current
achievement with regard to the conditional density, the student's growth
percentile situates the student's outcome at time \(t\) taking account
of past student performance. The percentile result reflects the
likelihood of such an outcome given the student's prior achievement. In
the sense that the student growth percentile translates to the
probability of such an outcome occurring (i.e., rarity), it is possible
to compare the progress of individuals not beginning at the same
starting point. However, occurrences being equally rare does not
necessarily imply that they are equally ``good.'' Qualifying student
growth percentiles as ``(in)adequate,'' ``good,'' or as satisfying ``a
year's growth'' is a standard setting procedure requiring external
criteria (e.g., growth relative to state performance standards) combined
with the wisdom and judgments of stakeholders.

Estimation of the conditional density is performed using quantile
regression (Koenker, 2005). Whereas linear regression methods model the
conditional mean of a response variable \(Y\), quantile regression is
more generally concerned with the estimation of the family of
conditional quantiles of \(Y\). Quantile regression provides a more
complete picture of both the conditional distribution associated with
the response variable(s). The techniques are ideally suited for
estimation of the family of conditional quantile functions (i.e.,
reference percentile curves). Using quantile regression, the conditional
density associated with each student's prior scores is derived and used
to situate the student's most recent score. Position of the student's
most recent score within this density can then be used to characterize
the student's growth. Though many state assessments possess a vertical
scale, such a scale is not necessary to produce student growth
percentiles.

In analogous fashion to the least squares regression line representing
the solution to a minimization problem involving squared deviations,
quantile regression functions represent the solution to the optimization
of a loss function (Koenker, 2005). Formally, given a class of suitably
smooth functions, \(\cal{G}\), one wishes to solve

\begin{equation} \textit{arg min}_ {g \in \cal{G}} \sum_ {i=1}^n \rho_ {\tau} (Y(t_ i) - g(t_ i)),\end{equation}

where \(t_i\) indexes time, \(Y\) are the time dependent measurements,
and \(\rho_{\tau}\) denotes the piecewise linear loss function defined
by

\begin{multline} \rho_ {\tau} (u) = u \cdot (\tau - I(u < 0)) = \begin{cases} u \cdot \tau & u \geq 0 \\ u \cdot (\tau - 1) & u < 0.  \end{cases}\end{multline}

The elegance of the quantile regression Expression 1 can be seen by
considering the more familiar least squares estimators. For example,
calculation of \(\textit{arg min} \sum_ {i=1}^n (Y_ i - \mu)^2\) over
\(\mu \in \mathbb{R}\) yields the sample mean. Similarly, if
\(\mu(x) = x^{\prime} \beta\) is the conditional mean represented as a
linear combination of the components of \(x\), calculation of
\(\textit{arg min} \sum_ {i=1}^n (Y_ i - x_ i^{\prime} \beta)^2\) over
\(\beta \in \mathbb{R}^p\) gives the familiar least squares regression
line. Analogously, when the class of candidate functions \(\cal{G}\)
consists solely of constant functions, the estimation of Expression 1
gives the \(\tau\)\(^{th}\) sample quantile associated with \(Y\). By
conditioning on a covariate \(x\), the \(\tau\)\(^{th}\) conditional
quantile function is given by

\begin{equation} Q_y (\tau | x) = \textit{arg min}_{\beta \in \mathbb{R}^{^p}} \sum_{i=1}^n \rho_{\tau} (y_i - x_i^{\prime} \beta).\end{equation}

In particular, if \(\tau=0.5\), then the estimated conditional quantile
line is the median regression line.\footnote{For a detailed treatment of
  the procedures involved in solving the optimization problem associated
  with Expression 1, see (Koenker, 2005), particularly Chapter 6.}

Following Wei and He (2006), we parameterize the conditional quantile
functions as a linear combination of B-spline cubic basis functions.
B-splines are employed to accommodate non-linearity, heteroscedasticity
and skewness of the conditional densities associated with values of the
independent variable(s). B-splines are attractive both theoretically and
computationally in that they provide excellent data fit, seldom lead to
estimation problems (Harrell, 2001), and are simple to implement in
available software.

Figure B.2 gives a bivariate representation of linear and B-splines
parameterization of decile growth curves. The assumption of linearity
imposes conditions upon the heteroscedasticity of the conditional
densities. Close examination of the linear deciles indicates slightly
greater variability for higher grade 5 scale scores than for lower
scores. By contrast, the B-spline based decile functions better capture
the greater variability at both ends of the scale score range together
with a slight, non-linear trend to the data.

\begin{figure}[H]
\caption*{\label{fig:quantPlot} {\bf{Fig. B.2:}} Linear and B-spline conditional deciles based upon bivariate math data, grades 5 and 6.}
  \begin{subfigure}[b]{0.5\textwidth}
    \includegraphics[width=\textwidth]{../img/Appendices/SGP_Method/linearquantileplot.png}
  \end{subfigure}
  %
  \begin{subfigure}[b]{0.5\textwidth}
    \includegraphics[width=\textwidth]{../img/Appendices/SGP_Method/bsplinequantileplot.png}
  \end{subfigure}
\end{figure}

Calculation of student growth percentiles is performed using \texttt{R}
(R Development Core Team, 2015), a language and environment for
statistical computing, with \texttt{SGP} package (Betebenner,
{VanIwaarden}, Domingue, \& Shang, 2014). Other possible software
(untested with regard to student growth percentiles) with quantile
regression capability include SAS and Stata. Estimation of cohort
referenced student growth percentiles is conducted using all available
prior data, subject to certain suitability conditions. Given assessment
scores for \(t\) occasions, (\(t \geq 2\)), the \(\tau\)\(^{th}\)
conditional quantile for \(Y_ t\) based upon
\(Y_ {t-1}, Y_ {t-2}, \ldots, Y_1\) is given by

\begin{equation} Q_ {Y_ t} (\tau | Y_ {t-1}, \ldots, Y_ 1) = \sum_ {j=1}^{t-1} \sum_ {i=1}^3 \phi_ {ij}(Y_ j)\beta_ {ij}(\tau),\end{equation}

where \(\phi_ {i,j}\), \(i=1,2,3\) and \(j=1, \ldots, t-1\) denote the
B-spline basis functions. Currently, bases consisting of 7 cubic
polynomials are used to ``smooth'' irregularities found in the
multivariate assessment data. A bivariate rendering of this is found is
Figure B.2 where linear and B-spline conditional deciles are presented.
The cubic polynomial B-spline basis functions model the
heteroscedasticity and non-linearity of the data to a greater extent
than is possible using a linear parameterization.

The B-spline basis functions require the selection of boundary and
interior knots. Boundary knots are end points outside of the scale score
distribution that anchor the B-spline basis. These are generally
selected by extending the range of scale scores by 10\%. That is, they
are defined as lying 10\% below the lowest obtainable (or observed)
scale score (LOSS) and 10\% above the highest obtainable scale score
(HOSS). The interior knots are the \emph{internal} breakpoints that
define the spline.

The default choice in the \texttt{SGP} package (Betebenner et al., 2014)
is to select the 20\(^{th}\), 40\(^{th}\), 60\(^{th}\) and 80\(^{th}\)
quantiles of the observed scale score distribution. In general the knots
and boundaries are computed using a distribution from several years of
compiled test data (i.e. multiple cohorts) so that any irregularities in
a single year are smoothed out. Subsequent annual analyses then use
these same knots and boundaries as well. All defaults were used to
compile the knots and boundaries for Utah from the CRT tests. New knots
and boundaries will be required for SAGE assessments beginning in 2015
when they will be used as the most recent dependent variables in the
quantile regressions.

Finally, it should be noted that the independent estimation of the
regression functions can potentially result in the crossing of the
quantile functions. This occurs near the extremes of the distributions
and is potentially more likely to occur given the use of non-linear
functions. The result of allowing the quantile functions to cross in
this maner would be \emph{lower} percentile estimations of growth for
\emph{higher} observed scale scores at the extremes (give all else equal
in prior scores) and vice versa. In order to deal with these
contradictory estimates, quantile regression results are isotonized to
prevent quantile crossing following the methods derived by Chernozhukov,
Fernandez-Val and Glichon (2010).

\pagebreak

\section{Discussion of Model
Properties}\label{discussion-of-model-properties}

Student growth percentiles possess a number of attractive properties
from both a theoretical as well as a practical perspective. Foremost
among practical considerations is that the percentile descriptions are
familiar and easily communicated to teachers and other non-technical
stakeholders. Furthermore, implicit within the percentile quantification
of student growth is a statement of probability. Questions of ``how much
growth is enough?'' or ``how much is a year's growth?'' ask stakeholders
to establish growth percentile thresholds deemed adequate. These
thresholds establish growth standards that translate to probability
statements. In this manner, percentile based growth forms a basis for
discussion of rigorous yet attainable growth standards for all children
supplying a norm-referenced context for Linn's existence proof (Linn,
2003) with regard to student level growth.

In addition to practical utility, student growth percentiles possess a
number of technical attributes well suited for use with assessment
scores. The more important theoretical properties of growth percentiles
include:

\begin{itemize}
\itemsep1pt\parskip0pt\parsep0pt
\item
  \textbf{Robustness to outliers.} Estimation of student growth
  percentiles are more robust to outliers than is traditionally the case
  with conditional mean estimation. Analogous to the property of the
  median being less influenced by outliers than is the median,
  conditional quantiles are robust to extreme observations. This is due
  to the fact that influence of a point on the \(\tau\)\(^{th}\)
  conditional quantile function is not proportional (as is the case with
  the mean) to the distance of the point from the quantile function but
  only to its position above or below the function (Koenker, 2005, p.
  44).
\item
  \textbf{Uncorrelated with prior achievement.} Analogous to least
  squares derived residuals being uncorrelated with independent
  variables, student growth percentiles are not correlated with prior
  achievement. This property runs counter to current multilevel
  approaches to measuring growth with testing occasion nested within
  students (Singer \& Willett, 2003). These models, requiring a vertical
  scale, fit lines with distinct slopes and intercepts to each student.
  The slopes of these lines represent an average rate of increase,
  usually measured in scale score points per year, for the student.
  Whereas a steeper slope represents more learning, it is important to
  understand that using a norm-referenced quantification of growth, one
  cannot necessarily infer that a low achieving student with a growth
  percentile of 60 ``learned as much'' as a high achieving student with
  the same growth percentile. Growth percentiles bypass questions
  associated with magnitude of learning and focus on norm-referencedly
  quantifying changes in achievement.\\
\item
  \textbf{Equivariance to monotone transformation of scale.} An
  important attribute of the quantile regression methodology used to
  calculate student growth percentiles is their invariance to monotone
  transformations of scale. This property, denoted by (Koenker, 2005) as
  \emph{equivariance to monotone transformations} is particularly
  helpful in educational assessment where a variety of scales are
  present for analysis, most of which are related by some monotone
  transformation. For example, it is a common misconception that one
  needs a vertical scale in order to calculate growth. Because vertical
  and non-vertical scales are related via a monotone transformation, the
  student growth percentiles do not change given such alterations in the
  underlying scale. This result obviates much of the discussion
  concerning the need for a vertical scale in measuring
  growth.\footnote{As already noted with regard to pediatrics, the
    existence of nice ``vertical'' scales for measuring height and
    weight still leads to observed changes being normed.}
\end{itemize}

Formally, given a monotone transformation \(h\) of a random variable
\(Y\),

\begin{equation} Q_ h(Y)|X (\tau | X) = h(Q_ Y|X (\tau | X)).\end{equation}

This result follows from the fact that
\(\Pr (T < t | X) = \Pr (h(T) < h(t) | X)\) for monotone \(h\). It is
important to note that \emph{equivariance to monotone transformation}
does not, in general, hold with regard to least squares estimation of
the conditional mean. That is, except for affine transformations \(h\),
\(E(h(Y)|X) \not= h(E(Y|X))\). Thus, analyses built upon mean based
regression methods are, to an extent, scale dependent.

\pagebreak

\section*{References}\label{references}
\addcontentsline{toc}{section}{References}

Ballou, D., Sanders, W., \& Wright, P. (2004). Controlling for student
background in value-added assessment for teachers. \emph{Journal of
Educational and Behavioral Statistics}, \emph{29}(1), 37--65.

Betebenner, D. W. (2008). Toward a normative understanding of student
growth. In K. E. Ryan \& L. A. Shepard (Eds.), \emph{The future of
test-based educational accountability} (pp. 155--170). New York: Taylor
\& Francis.

Betebenner, D. W. (2009). Norm- and criterion-referenced student growth.
\emph{Educational Measurement: Issues and Practice}, \emph{28}(4),
42--51.

Betebenner, D. W., {VanIwaarden}, A., Domingue, B., \& Shang, Y. (2014).
\emph{SGP: An r package for the calculation and visualization of student
growth percentiles \& percentile growth trajectories.} Retrieved from
\url{https://github.com/CenterForAssessment/SGP}

Braun, H. I. (2005). \emph{Using student progress to evaluate teachers:
A primer on value-added models}. Princeton, New Jersey: Educational
Testing Service.

Chernozhukov, V., Fern{á}ndez-Val, I., \& Galichon, A. (2010). Quantile
and probability curves without crossing. \emph{Econometrica},
\emph{78}(3), 1093--1125. Wiley Online Library.

Harrell, F. E. (2001). \emph{Regression modeling strategies}. New York:
Springer.

Harris, D. N. (2007). \emph{The policy uses and ``policy validity'' of
value-added and other teacher quality measures}. Princeton, NJ:
Educational Testing Service.

Koenker, R. (2005). \emph{Quantile regression}. Cambridge: Cambridge
University Press.

Linn, R. L. (2003). \emph{Accountability: Responsibility and reasonable
expectations}. Los Angeles, CA: Center for the Study of Evaluation,
CRESST.

Linn, R. L., Baker, E. L., \& Betebenner, D. W. (2002). Accountability
systems: Implications of requirements of the No Child Left Behind Act of
2001. \emph{Educational Researcher}, \emph{31}(6), 3--16.

Lord, F. M. (1975). The ``ability'' scale in item characteristic curve
theory. \emph{Psychometrika}, \emph{20}, 299--326.

R Development Core Team. (2015). \emph{R: A language and environment for
statistical computing}. Vienna, Austria: R Foundation for Statistical
Computing. Retrieved from \url{http://www.R-project.org}

Raudenbush, S. W. (2004). What are value-added models estimating and
what does this imply for statistical practice? \emph{Journal of
Educational and Behavioral Statistics}, \emph{29}(1), 121--129.

Rubin, D. B., Stuart, E. A., \& Zanutto, E. L. (2004). A potential
outcomes view of value-added assessment in education. \emph{Journal of
Educational and Behavioral Statistics}, \emph{29}(1), 103--116.

Singer, J. D., \& Willett, J. B. (2003). \emph{Applied longitudinal data
analysis}. New York: Oxford University Press.

Spellings, M. (2005). \emph{Secretary Spellings Announces Growth Model
Pilot}. Press Release, U.S. Department of Education.

Wei, Y., \& He, X. (2006). Conditional growth charts. \emph{The Annals
of Statistics}, \emph{34}(5), 2069--2097.

Yen, W. M. (1986). The choice of scale for educational measurement: An
IRT perspective. \emph{Journal of Educational Measurement}, \emph{23},
299--325.

Yen, W. M. (2007). Vertical scaling and No Child Left Behind. In N. J.
Dorans, M. Pommerich, \& P. W. Holland (Eds.), \emph{Linking and
aligning scores and scales} (pp. 273--283). New York: Springer.

\bibliographystyle{plainnat}
\bibliography{/Library/Frameworks/R.framework/Versions/3.2/Resources/library/SGPreports/rmarkdown/templates/multi_document/resources/educ.bib}




\end{document}


