\documentclass[12pt]{article}
\usepackage[margin=0.915in]{geometry}

%%%
%%%             BEGIN Damian Preamble
%%%   Preface commands for pdfLaTeX output
%%%

%%
%% Calling relevant packages
%%

\usepackage[%
pdfstartview={FitV}, 
colorlinks=true,
menucolor=DarkGray,
linkcolor=MidnightBlue,
citecolor=MidnightBlue,
urlcolor=OrangeRed]{hyperref}
\usepackage{graphicx}
\DeclareGraphicsExtensions{.pdf}

%% From pandoc default: Begin
\usepackage{graphicx}
\makeatletter
\def\maxwidth{\ifdim\Gin@nat@width>\linewidth\linewidth\else\Gin@nat@width\fi}
\def\maxheight{\ifdim\Gin@nat@height>\textheight\textheight\else\Gin@nat@height\fi}
\makeatother
% Scale images if necessary, so that they will not overflow the page
% margins by default, and it is still possible to overwrite the defaults
% using explicit options in \includegraphics[width, height, ...]{}
\setkeys{Gin}{width=\maxwidth,height=\maxheight,keepaspectratio}
 %% From pandoc default: Begin

%\usepackage[hyper]{apacite}
\usepackage[svgnames]{xcolor}
\usepackage{rotating,bm,amsmath,amsfonts,amssymb,indentfirst,lscape,fancybox,fancyvrb,listings,pdfpages}
\usepackage[pagestyles]{titlesec}
%\usepackage{ucs}

%%
%% Format changes for chapter and section commands
%%

%\input{onesidedheader.tex}

%%
%% Chapter declarations
%%


%%
%% Section declarations
%%

%%
%% Subsection declarations
%%


%%
%% Subsubsection declarations
%%

%%
%% Listings setup information
%%

\lstset{language=R, frame=ltrb, framesep=5pt, xleftmargin=12pt, xrightmargin=5pt,
       numbers=none, breaklines=true, fancyvrb=true,
       breakatwhitespace=true, captionpos=b, abovecaptionskip=1.5ex,
       backgroundcolor=\color{Cornsilk},
       basicstyle=\small\color{DarkSlateGrey},
       keywordstyle=\ttfamily\color{DarkSlateGrey},
       identifierstyle=\ttfamily\color{DarkSlateGrey}\bfseries, 
       commentstyle=\color{Fuchsia},
       stringstyle=\ttfamily,
       showstringspaces=false}

%%
%% Header and Footer Specification 
%% TO ACTIVATE THE HEADER/FOOTER, ONE MUST PLACE \pagestyle{plain} 
%% followed by \pagestyle{damian} in the document
%%

\widenhead{0.14in}{0.14in}

\renewpagestyle{plain}{}

\newpagestyle{damian}[\sffamily]{
\headrule
\sethead[\sectiontitle][\chaptertitle][\thepage]
  {\sectiontitle}{\chaptertitle}{\thepage}

\footrule
\setfoot[][\raisebox{-.85ex}[0pt]{\NavigationBar}][]
{}{\raisebox{-.85ex}[0pt]{\NavigationBar}}{}
	\newcommand{\NavigationBar}{%
	  \Acrobatmenu{PrevPage}{Previous}\hspace{.5cm}
	  \Acrobatmenu{NextPage}{Next}\hspace{.5cm}
	  \Acrobatmenu{FirstPage}{First}\hspace{.5cm}
	  \Acrobatmenu{LastPage}{Last}\hspace{.5cm}
	  \Acrobatmenu{GoBack}{Back}\hspace{.5cm}
	  \Acrobatmenu{Quit}{Quit}%
}
}

%%
%% Custom specifications, commands, and colors
%%

\definecolor{DarkGray}{cmyk}{0,0,0,.624}
\newcommand{\R}{{\sffamily \textup{R}}}
\newcommand{\MlwiN}{{\sffamily \textup{MlwiN}}}
\newcommand{\Sweave}{{\sffamily \textup{Sweave}}}
\newcommand{\sssty}{\scriptscriptstyle}
\newcommand{\bigdot}{\ensuremath{\hspace{-.3ex}\bm{.}\hspace{-.05ex}}}
\renewcommand{\abstractname}{Abstract}
\renewcommand{\lstlistingname}{\R~Code Example}
\renewcommand{\arraystretch}{1.2}
\setlength{\fboxsep}{4mm}
\setlength{\fboxrule}{0.4pt}

%%
%%		END Damian Preamble
%%

\hypersetup{%
  pdftitle={ The Utah Student Growth Model },
  pdfauthor={  Damian W. Betebenner  ,  Adam R. VanIwaarden  ,  \emph{National Center for the Improvement \ of Educational Assessment (NCIEA)}  },
  pdfcreator={ Damian W. Betebenner  , Adam R. VanIwaarden  , \emph{National Center for the Improvement \ of Educational Assessment (NCIEA)}  },
  pdfkeywords={},
  bookmarks=true} % pdfproducer={pdfLaTeX}

\usepackage{caption}
\usepackage{float}
\usepackage{longtable}
\usepackage{booktabs}
\usepackage{subcaption}
\usepackage{dcolumn}

\setcounter{secnumdepth}{3}




%\usepackage{pdfdraftcopy}
\newtheorem{proposition}{Proposition}
\newtheorem{theorem}{Theorem}
\newtheorem{definition}{Definition}
\newtheorem{corollary}{Corollary}
\DeclareMathOperator*{\argmin}{arg\,min}

% \usepackage{bbm}
\DeclareMathAlphabet{\mathbbm}{U}{bbm}{m}{n}
\SetMathAlphabet\mathbbm{bold}{U}{bbm}{bx}{n}
\DeclareMathAlphabet{\mathbbmss}{U}{bbmss}{m}{n}
\SetMathAlphabet\mathbbmss{bold}{U}{bbmss}{bx}{n}
\DeclareMathAlphabet{\mathbbmtt}{U}{bbmtt}{m}{n}


\newcommand{\pl}[1]{\textsf{PL#1}}
\newcommand{\Cov}{\ensuremath{\mbox{\textsf{Cov}}}}
\newcommand{\Diag}{\ensuremath{\mbox{\textsf{Diag}}}}
\newcommand{\Bias}{\ensuremath{\mbox{\textsf{Bias}}}}
\newcommand{\Astar}[1]{\ensuremath{#1^{^*}}}
\thispagestyle{plain}
\pagestyle{damian}

\begin{document}

\title{\textsf{\LARGE The Utah Student Growth Model  \\\medskip A Technical Overview of the 2014-2015 Student Growth Percentile
Calculations }}
\author{  Damian W. Betebenner    \\   Adam R. VanIwaarden    \\   \emph{National Center for the Improvement \ of Educational Assessment (NCIEA)}   }

 \date{October 2015} 

\maketitle

\newpage

\begin{abstract}
DRAFT REPORT - DO NOT CITE! This report provides details about the Utah
Student Growth Model methodology and presents a descriptive analysis of
the 2015 SGP results.
\end{abstract}
\newpage

\section{Introduction}\label{introduction}

This report contains details on the implementation of the student growth
percentiles (SGP) model for the state of Utah. The National Center for
the Improvement of Educational Assessment (NCIEA) contracted with the
Utah State Office of Education (USOE) to implement the SGP methodology
using data derived from the Utah student assessment program to create
the Utah Student Growth Model. The goal of the engagement with USOE is
to create a set of open source analytic techniques and conduct a set of
initial analyses that will eventually be conducted by USOE in following
years.

The SGP methodology is an open source norm- and criterion-referenced
student growth analysis that produces student growth percentiles and
student growth projections/targets for each student with longitudinal
data in the state. The methodology is currently used for many purposes.
States and districts have used the results in various ways including
parent/student diagnostic reporting, institutional improvement, and
school and educator accountability.

The report includes four sections covering Data, Analytics, SGP Results,
and Goodness of Fit:

\begin{itemize}
\itemsep1pt\parskip0pt\parsep0pt
\item
  \emph{Data} includes details on the decision rules used in the raw
  data preparation and student record validation.
\item
  \emph{Analytics} introduces some of the basic methodological concepts
  implemented.
\item
  \emph{SGP Results} provides basic descriptive statistics from the 2015
  analyses.
\item
  \emph{Goodness of Fit} describes how well the statistical models used
  to produce SGPs fit Utah students' data. This includes discussion of
  goodness of fit plots and the student- and school level correlations
  between SGP and prior achievement.
\end{itemize}

Additionally, multiple appendices are included. Appendix A includes
Goodness of Fit plots for all content areas and grades. Appendix B
provides a more technical description of the SGP methodology and
statistical concepts.

\pagebreak

\section{Data}\label{data}

Data for the Utah Student Assessment of Growth and Excellence (SAGE)
used in the SGP analyses were supplied by the Utah SOE to the NCIEA for
analysis in August 2015. The current longitudinal data set now includes
academic years 2007-2008 through 2014-2015. Subsequent years' analyses
will augment this multi-year data set allowing USOE to maintain a
comprehensive longitudinal data set for all students taking the SAGE and
EOCT assessments.

Student Growth Percentiles have been produced for students that have a
current score and at least one prior score in the same subject or a
related content area. SGPs were produced for English Language Arts
(ELA), Science and Mathematics end-of-grade tests (EOGT). For the 2015
academic year SGPs were produced for end-of-course tests (EOCT) for
Earth Science, Biology, Chemistry, Physics, Secondary Math I, Secondary
Math II and Secondary Math III courses.

\subsection{Longitudinal Data}\label{longitudinal-data}

Growth analyses on assessment data require data which are linked to
individual students over time. Student growth percentile analyses
require, at a minimum, two years of assessment data for analysis of
student progress. To this end it is necessary that a unique student
identifier be available so that student data records across years can be
merged with one another and subsequently examined. Because some records
in the assessment data set contain students with more than one test
score in a content area in a given year, a process to create unique
student records in each content area by year combination is required in
order to carry out subsequent growth analyses. The following business
rules were used to select the appropriate record for use in the
analyses.

\subsubsection{Student Record Selection Business
Rules}\label{student-record-selection-business-rules}

\begin{enumerate}
\def\labelenumi{\arabic{enumi}.}
\itemsep1pt\parskip0pt\parsep0pt
\item
  Student records with missing (``NA'') scores or scale scores outside
  of the possible range are invalidated.
\item
  If a student has multiple records (duplicate from the same subject and
  grade), the score associated with the most advance course is selected.
\item
  Students who are not enrolled in a school for the full academic year
  (FAY) are omitted from the production of model coefficient matrices.
  Unlike previous years, these students do have SGPs created
  subsequently using the model coefficient matrices produced using
  FAY/accountability eligible data.
\item
  Non-standard participation or accommodation students are omitted.
\end{enumerate}

Table 1 shows the number of valid grade level student records available
for analysis\footnote{This number does not represent the number of SGPs
  produced, however, because students are required to have at least one
  prior score available as well.} and Table 2 shows the available number
of valid EOCT records.

\begin{table}[H]
\caption*{\textbf{Table 1:} Number of Valid Grade Level Student Records by Grade and Subject for 2015\label{table1}} 
\begin{center}
\begin{tabular}{lclllllllll}
\hline\hline
\multicolumn{1}{c}{\bfseries }&\multicolumn{1}{c}{\bfseries }&\multicolumn{9}{c}{\bfseries Grades}\tabularnewline
\cline{1-11}
\multicolumn{1}{c}{Content Area}&\multicolumn{1}{c}{}&\multicolumn{1}{c}{3}&\multicolumn{1}{c}{4}&\multicolumn{1}{c}{5}&\multicolumn{1}{c}{6}&\multicolumn{1}{c}{7}&\multicolumn{1}{c}{8}&\multicolumn{1}{c}{9}&\multicolumn{1}{c}{10}&\multicolumn{1}{c}{11}\tabularnewline
\hline
ELA&&48,206&46,913&46,839&46,121&44,415&43,577&42,167&40,560&37,438\tabularnewline
Mathematics&&48,500&47,139&47,103&46,169&43,617&43,487&&&\tabularnewline
Science&&&47,141&47,149&46,332&44,783&44,033&&&\tabularnewline
\hline
\end{tabular}\end{center}

\end{table}

\begin{table}[H]
\caption*{\textbf{Table 2:} Total Number of Valid EOCT Student Records by Subject for 2015\label{table2}} 
\begin{center}
\begin{tabular}{ll}
\hline\hline
\multicolumn{1}{c}{Content Area}&\multicolumn{1}{c}{}\tabularnewline
\hline
Earth Science&25,473\tabularnewline
Biology&43,922\tabularnewline
Chemistry&24,403\tabularnewline
Physics&18,704\tabularnewline
Sec Math I&44,955\tabularnewline
Sec Math II&41,270\tabularnewline
Sec Math III&28,302\tabularnewline
\hline
\end{tabular}\end{center}

\end{table}

\pagebreak

\section{Analytics}\label{analytics}

This section provides basic details about the calculation of student
growth percentiles and percentile growth trajectories (`projections')
from Utah state assessment data using the
\href{http://www.r-project.org/}{\texttt{R} Software Environment} (R
Development Core Team, 2015) in conjunction with the
\href{https://github.com/CenterForAssessment/SGP}{\texttt{SGP} Package}
(Damian W. Betebenner, {VanIwaarden}, Domingue, \& Shang, 2015). More in
depth treatment of the data analysis process with code examples is
available to USOE the staff through Github.

Broadly, the SGP analysis of the Utah longitudinal student assessment
data takes place in two steps:

\begin{enumerate}
\def\labelenumi{\arabic{enumi}.}
\itemsep1pt\parskip0pt\parsep0pt
\item
  Data Preparation
\item
  Data Analysis
\end{enumerate}

Those familiar with data analysis know that the bulk of the effort in
the above two step process lies with Step 1: Data Preparation. Following
thorough data cleaning and preparation, data analysis using the
\texttt{SGP} Package takes clean data and makes it as easy as possible
to calculate, summarize, output and visualize the results from SGP
analyses.

\subsection{Data Preparation}\label{data-preparation}

The data preparation step involves taking data provided by the USOE and
producing a \texttt{.Rdata} file that will subsequently be analyzed in
Step 2. This process is carried out annually as new data becomes
available from the state assessment program. The data supplied by the
USOE Information Technology department and subsequently cleaned and
processed using \texttt{R}.

In previous years' analyses the bulk of the data cleaning and
implementation of business rule validation was also performed in
\texttt{R} by NCIEA staff. However, the division of the cleaning and
validation tasks used in 2015 is an important step in moving towards
USOE self-sufficiency in calculating growth percentiles in subsequent
years.

The cleaned and formatted data was combined with the existing data used
up through the 2014 analyses. With an appropriate longitudinal data
prepared, we continued to the calculation of student-level SGPs.

\subsection{2015 Data Analysis}\label{data-analysis}

The objective of the student growth percentile (SGP) analysis is to
produce a measure which describes how (a)typical a student's growth is
by examining his/her current achievement relative to students with a
similar achievement history; i.e.~his/her \emph{academic peers}. The
estimation of this norm-referenced growth quantity is conducted using
quantile regression (Koenker, 2005) to model curvilinear functional
relationships between student's prior and current scores. One hundred
such regression calculations are run for each separate analysis (defined
as a unique year, content area, and grade combination). The end product
of these 100 separate regression models is a single coefficient matrix,
which serves as a look-up table to relate prior student achievement to
current achievement for each percentile. This process ultimately leads
to the calculation of thousands of calculations for each of Utah's
annual analyses. For a more in-depth discussion of the calculation and
estimation of SGPs, see Betebenner ((2009)) and Appendix B of this
report for further information on the SGP methodology.

The 2015 Utah SGP analyses follow a work flow established in previous
years that includes the following 4 steps:

\begin{enumerate}
\def\labelenumi{\arabic{enumi}.}
\itemsep1pt\parskip0pt\parsep0pt
\item
  Create annual SGP configurations for End-of-Grade Test (EOGT) and
  End-of-Course Test(EOCT) analyses.
\item
  Update the \texttt{SGPstateData} object in the \texttt{SGP} package.
  For 2015 this includes \textbf{\emph{a)}} updating the norm group
  preferences, \textbf{\emph{b)}} adding meta-data used in the
  calculation of student growth percentiles and projections, including
  the SAGE test knots and boundaries for the cubic basis splines and the
  test specific proficiency cutscores, and \textbf{\emph{c)}} adding new
  SAGE related meta-data for the production of individual student
  reports (ISRs).
\item
  Conduct EOGT and EOCT SGP Analyses.
\item
  Export data, and produce summaries and visualizations from the
  \texttt{Utah\_SGP} data object (including ISRs).
\end{enumerate}

\subsubsection{Create annual SGP
configurations.}\label{create-annual-sgp-configurations.}

The EOCT analyses are specialized enough so that it is necessary to
explicitly specify the analyses to be performed via a configuration code
script. For several years, configurations have been employed to conduct
EOCT SGP analyses for Utah. Unlike previous years, where EOGT analyses
were run separately and did not require configuration scripts, the 2015
analyses specify the configurations for these analyses. This allows the
analyses to be run more efficiently.

Each configuration specifies a set of parameters that defines the norm
group of students to be examined. Every potential norm group is defined
by, at a minimum, the progressions of content area, academic year and
grade-level. Other parameters may also be defined. Each configuration
used for the Utah EOCT analyses contain the first three elements. The
EOCT analyses also contain the fourth and fifth elements:

\begin{itemize}
\itemsep1pt\parskip0pt\parsep0pt
\item
  \textbf{\texttt{sgp.content.areas}:} A progression of values that
  specifies the content areas to be looked at and their order
\item
  \textbf{\texttt{sgp.panel.years}:} The progression of the years
  associated with the content area progression
  (\texttt{sgp.content.areas}) provided in the configuration,
  potentially allowing for skipped years, etc.
\item
  \textbf{\texttt{sgp.grade.sequences}:} The grade progression
  associated with the content area and year progressions provided in the
  configuration. \textbf{`EOCT'} stands for `End Of Course Test'. The
  use of the generic `EOCT' allows for secondary students to be compared
  based on the pattern of course taking rather than being dependent upon
  grade-level/class-designation.
\item
  \textbf{\texttt{sgp.projection.grade.sequences}:} This element is used
  to identify the configurations that will be used to produce straight
  and/or lagged student growth projections. It can, somewhat
  counter-intuitively, be left out or set to NULL, in which case
  projections will be produced and the package functions will populate
  the grade sequences to use based on the values provided in the
  \texttt{sgp.grade.sequences} element. Alternatively, when set to
  ``NO\_PROJECTIONS'', no projections will be produced. For EOCT
  analyses, only configurations that correspond to the canonical course
  progressions can produce straight or lagged student growth
  projections. The canonical progressions are codified in
  \texttt{SGPstateData{[}{[}"UT"{]}{]}{[}{[}"SGP\_Configuration"{]}{]}{[}{[}"content\_area.projection.sequence"{]}{]}}.
\item
  \textbf{\texttt{sgp.norm.group.preference}:} Because a student can
  potentially be included in more than one analysis/configuration, this
  argument provides a ranking specifying which SGP is preferable and
  will ultimately be the SGP matched with the student in the
  \href{https://github.com/CenterForAssessment/SGP/blob/master/R/combineSGP.R}{\texttt{combineSGP}}
  step. \textbf{\emph{Lower numbers correspond with higher preference.}}
\end{itemize}

The \texttt{sgp.content.areas}, \texttt{sgp.panel.years}, and
\texttt{sgp.grade.sequences} elements all correspond to values found in
the Utah data, and are used to select the subset of the longitudinal
data set to be analyzed. Only these three elements are needed for the
EOGT analyses because they automatically fall into canonical projection
sequences and there will not be any duplicates produced given the data
cleaning process.

\subsubsection{Update Utah assessment
meta-data}\label{update-utah-assessment-meta-data}

The use of higher-level functions included in the SGP package (e.g.
\texttt{analyzeSGP}) requires the availability of state specific
assessment information. This meta-data is compiled in a \texttt{R}
object named \texttt{SGPstateData} that is housed in the package. The
required updates for the 2015 analyses included \textbf{\emph{a)}}
updating the norm group preferences, \textbf{\emph{b)}} adding meta-data
used in the calculation of student growth percentiles and projections,
including the SAGE test knots and boundaries for the cubic basis splines
and the test specific proficiency cutscores, and \textbf{\emph{c)}}
adding new SAGE related meta-data for the production of individual
student reports (ISRs).

\textbf{Norm group preferences}

The process through which EOCT analyses are run can produce multiple
SGPs for some students. In order to identify which quantity will be used
as the students' ``official'' SGP and subsequently merged into the
master longitudinal data set, a system of norm group preferencing is
established and is encoded into a look-up table and included in the
\texttt{SGPstateData}. In general, the preference is given to:

\begin{itemize}
\itemsep1pt\parskip0pt\parsep0pt
\item
  Progressions with the greatest number of prior scale scores.
\item
  Progressions in which a student has repeated a course.
\item
  Progressions that do not include a skipped year (i.e.~a gap in the
  scale score history).
\item
  Progressions which fall into the ``canonical'' course progression, on
  which student growth projection estimates are produced and adequate
  growth judgement \emph{potentially} made.
\end{itemize}

\textbf{Calculation and addition of knots and boundaries}

Calculation of SGPs includes the use of cubic B-spline basis functions
to more adequately model the heteroscedasticity and non-linearity found
in assessment data.\footnote{It should be noted that the independent
  estimation of the regression functions can potentially result in the
  crossing of the quantile functions. This occurs near the extremes of
  the distributions and is potentially more likely to occur given the
  use of non-linear functions. A potential result of allowing the
  quantile functions to cross would be \emph{lower} percentile
  estimations of growth for \emph{higher} observed scale scores at the
  extremes (give all else equal in students' prior score histories) and
  vice versa. In order to deal with these contradictory estimates,
  quantile regression results are isotonized to prevent quantile
  crossing following the methods derived by Chernozhukov, Fernandez-Val
  and Glichon (2010).} These functions require the selection of boundary
and interior knots. Boundary knots are end-points outside of the scale
score distribution that anchor the B-spline basis. These are generally
selected by extending the entire range of scale scores by 10\%. That is,
they are defined as lying 10\% below the lowest obtainable (or observed)
scale score (LOSS) and 10\% above the highest obtainable scale score
(HOSS). The interior knots are the \emph{internal} breakpoints that
define the spline. The default choice in the \texttt{SGP} package
(Damian W. Betebenner et al., 2015) is to select the 20\(^{th}\),
40\(^{th}\), 60\(^{th}\) and 80\(^{th}\) quantiles of the observed scale
score distribution.

In general the knots and boundaries are computed from a distribution
comprised of several years of test data (i.e.~multiple cohorts) so that
any irregularities in a single year are smoothed out. This is important
because subsequent annual analyses use these same knots and boundaries
as well. All defaults were used to compile the knots and boundaries for
Utah from the CRT and EOCT tests in previous years, and were also used
in 2015 to compute the SAGE assessments' knots and boundaries using SAGE
data from 2014 and 2015. New knots and boundaries will be required for
Utah SAGE assessments beginning in 2015 as they are now used as the
dependent variables in the quantile regressions.

\textbf{SAGE proficiency level cutscores}

Cutscores, which are set externally by the USOE through standard-setting
processes, are mainly required for student growth projections. These
growth projection estimates are used in the computation of adequate
growth measures and elements of the ISRs. The SAGE cutscore data was
provided to the Center for Assessment staff in 2014 and added at that
time. This year is the first year in which those values were used in the
analyses.

\textbf{ISR meta-data}

Finally, meta data for the ISR production was added. Mainly this
entailed updating the \texttt{Assessment\_Program\_Information} and
\texttt{Student\_Report\_Information} sections. The entire 2015 Utah
entry of the \texttt{SGPstateData} can be
\href{https://github.com/CenterForAssessment/SGPstateData/blob/2f1362d2411a94f7e244127cdb332683a3c2ba05/SGPstateData.R\#L4702}{viewed
here.}

\subsubsection{Conduct SGP Analyses.}\label{conduct-sgp-analyses.}

Unlike the 2014 analyses, we use the \texttt{updateSGP} function to A)
do the final preparation and addition of the new long data to the
existing \texttt{SGP} data object (t\texttt{prepareSGP} step) and B)
produce SGPs for \textbf{\emph{both}} the grade-level and EOCT subjects
(\texttt{analyzeSGP} step). Also, unlike any previous years' analyses,
we produced SGPs for students who were non-continuously enrolled for the
full academic year (FAY). Only the FAY students were used to construct
the estimating models and their associated coefficient matrices. The
non-FAY analyses utilize these model coefficient matrices to produce
those students' SGPs. Therefore, there are two analyses being run with
the data submitted this year by USOE (in strict order): \emph{1)} 2015
FAY students and \emph{2)} 2015 non-FAY students.

\subsubsection{Merge 2015 results into the longitudinal data, and
output, summarize and visualize
data.}\label{merge-2015-results-into-the-longitudinal-data-and-output-summarize-and-visualize-data.}

Once all analyses were completed the results were merged into the master
longitudinal data set. A pipe delimited version of the complete long
data is output and submitted to USOE. The data is also summarized using
the \texttt{summarizeSGP} function, which produced many tables of
descriptive statistics that are disaggregated at the state, district,
school and other factors of interest. Finally, visualizations (such as
bubble charts and ISRs) are produced from the data and summary tables.

\pagebreak

\section{SGP Results}\label{sgp-results}

The following sections provide basic descriptive statistics from the
2015 analyses, including the state level mean and median growth
percentiles. Currently Utah uses cohort referenced SGPs as the official
student level growth metric. The interested reader can find more in
depth discussions of the SGP methodology in Appendix B of this report.

\subsection{Median SGPs}\label{median-sgps}

Growth percentiles, being quantities associated with each individual
student, can be easily summarized across numerous grouping indicators to
provide summary results regarding growth. The median and mean of a
collection of growth percentiles are used as measures of central
tendency that summarize the distribution as a single number. With
perfect data fit, we expect the state median of all student growth
percentiles in any grade to be 50 because the data are norm-referenced
across all students in the state. Median (and mean) growth percentiles
well below 50 represent growth less than the state ``average'' and
median growth percentiles well above 50 represent growth in excess of
the state ``average''.

To demonstrate the norm-referenced nature of the growth percentiles
viewed at the state level, Table 3 presents the cohort-referenced growth
percentile medians and means for the EOGT content areas and 4 shows the
EOCT subjects.

\begin{table}[H]
\caption*{\textbf{Table 3:} Grade Level SAGE Median (Mean) Student Growth Percentile by Grade and Content Area for 2015\label{}} 
\begin{center}
\begin{tabular}{lcllllllll}
\hline\hline
\multicolumn{1}{c}{\bfseries }&\multicolumn{1}{c}{\bfseries }&\multicolumn{8}{c}{\bfseries Grades}\tabularnewline
\cline{1-10}
\multicolumn{1}{c}{Content Area}&\multicolumn{1}{c}{}&\multicolumn{1}{c}{4}&\multicolumn{1}{c}{5}&\multicolumn{1}{c}{6}&\multicolumn{1}{c}{7}&\multicolumn{1}{c}{8}&\multicolumn{1}{c}{9}&\multicolumn{1}{c}{10}&\multicolumn{1}{c}{11}\tabularnewline
\hline
ELA&&50 (49.8)&50 (49.8)&50 (49.8)&50 (49.7)&50 (49.9)&49 (49.5)&50 (49.7)&50 (49.7)\tabularnewline
Mathematics&&50 (49.7)&50 (49.8)&50 (49.9)&50 (49.8)&49 (49.5)&&&\tabularnewline
Science&&&50 (49.8)&50 (49.8)&50 (49.8)&50 (49.8)&&&\tabularnewline
\hline
\end{tabular}\end{center}

\end{table}

\begin{table}[H]
\caption*{\textbf{Table 4:} EOCT SAGE Median and Mean Student Growth Percentile by Content Area for 2015\label{}} 
\begin{center}
\begin{tabular}{lll}
\hline\hline
\multicolumn{1}{c}{Content Area}&\multicolumn{1}{c}{Median SGP}&\multicolumn{1}{c}{Mean SGP}\tabularnewline
\hline
Earth Science&49&49.4\tabularnewline
Biology&50&49.7\tabularnewline
Chemistry&50&50.0\tabularnewline
Physics&50&49.9\tabularnewline
Sec Math I&50&49.7\tabularnewline
Sec Math II&51&50.8\tabularnewline
Sec Math III&50&50.2\tabularnewline
\hline
\end{tabular}\end{center}

\end{table}

Based upon perfect model fit to the data, the median of all state growth
percentiles in each grade by year by subject combination should be 50.
That is, in the conditional distributions, 50 percent of growth
percentiles should be less than 50 and 50 percent should be greater than
50. Deviations from 50 indicate imperfect model fit to the data.
Imperfect model fit can occur for a number of reasons, some due to
issues with the data (e.g., floor and ceiling effects leading to a
``bunching'' up of the data) as well as issues due to the way that the
SGP function fits the data. The results in Table 3 and 4 are close to
perfect, with almost all values equal to 50.

The results are coarse in that they are aggregated across tens of
thousands of students. More refined fit analyses are presented in the
Goodness-of-Fit section that follows. Depending upon feedback from Utah
SOE, it may be desirable to tweak with some operational parameters and
attempt to improve fit even further. The impact upon the operational
results based on better fit is expected to be extremely minor.

It is important to note how, at the entire state level, the
\emph{norm-referenced} growth information returns little information on
annual trends due to its norm-reference nature. What the results
indicate is that a typical (or average) student in the state
demonstrates 50\(^{th}\) percentile growth. That is, ``typical
students'' demonstrate ``typical growth''. One benefit of the
norm-referenced results follows when subgroups are examined (e.g.,
schools, district, demographic groups, etc.) Examining subgroups in
terms of the median of their student growth percentiles, it is then
possible to investigate why some subgroups display lower/higher student
growth than others. Moreover, because the subgroup summary statistic
(i.e., the median) is composed of many individual student growth
percentiles, one can break out the result and further examine the
distribution of individual results.

\pagebreak

\section{Goodness of Fit}\label{goodness-of-fit}

Examination of goodness-of-fit was conducted by comparing the estimated
conditional density with the theoretical uniform density of the SGPs.
Despite the use of B-splines to accommodate heteroscedasticity and
skewness of the conditional density, assumptions are made concerning the
number and position of spline knots that impact the percentile curves
that are fit. With an infinite population of test takers, at each prior
scaled score, with perfect model fit, the expectation is to have 10
percent of the estimated growth percentiles between 1 and 9, 10 and 19,
20 and 29, \ldots{}, and 90 and 99. Deviations from 10 percent would be
indicative of lack of model fit.

\subsection{Model Fit Plots}\label{model-fit-plots}

Using all available grade level and EOCT scores as the variables,
estimation of student growth percentiles was conducted for each possible
student (those with a current score and at least one prior score). A
goodness of fit plot is produced for each unique analysis run in 2015.
Each analysis is defined by the grade and content area for the
grade-level analyses and the unique course progression sequences for the
end of course test (EOCT) subjects.

The ``Ceiling/Floor Effects Test'' panel is intended to help identify
potential problems in SGP estimation at the Highest and Lowest
Obtainable (or Observed) Scale Scores (HOSS and LOSS). Issues can occur
here where, when ceiling or floor effects are present in both the
current and prior year(s) scores, it may be relatively typical for
extremely high/low achieving students to consistently score at or near
the HOSS/LOSS. As a result, the SGPs for students scoring at the
HOSS/LOSS will be unexpectedly low/high. That is, for example, if a
sufficient number of students maintain performance at the HOSS over
time, this performance will be estimated to typical, and therfore SGP
estimates will reflect typical growth (e.g.~50th percentile). In some
cases these extreme score values or small deviations from them might
even yield low growth estimates. Although these score patterns can
ligitimately be estimated as a low growth percentiles because they
represent rather typical growth, it is potentially an unfair description
of their growth performance (and by proxy teacher, school or district
performance or ``value added'') caused by an artifact of the inability
of the assessment to adequately measure student performance at extreme
ability levels.

The table of values here shows whether the current year scale scores at
both extremes yield the expected SGPs\footnote{Note that the prior year
  scale scores are not represented here, but are critical in the SGP
  calculation of all students}. The expectation is that the majority of
SGPs for students scoring at or near the LOSS will be low (preferably
less than 5 and not higher than 10), and that SGPs for students scoring
at or near the HOSS will be high (preferably higher than 95 and not less
than 90). Because few students may score \emph{exactly} at the
HOSS/LOSS\footnote{This is particularly true when IRT Theta (\(\theta\))
  estimates are used rather than scaled scores, which often apply
  artificial LOSS/HOSS cut points.}, the top/bottom 50 students are
selected and any student scoring within their range of scores are
selected for inclusion in these tables. Consequently, there may be a
range of scores at the HOSS/LOSS rather than a single score\footnote{This
  can make the interpretation of the SGP distribution somewhat harder
  because score not directly at the extremes do not necessarily preclude
  maximum SGP estimates.}, and there may be more than 50 students
included in the HOSS/LOSS row if the 50 students at the extremes only
contain the single HOSS/LOSS score\footnote{This also leads to potential
  difficulties in interpretation because with a higher number of
  students comes a greater distribution of prior scale scores and
  therefore a greater distribution of the expected SGPs}. In either
case, a more fine grained analysis of the relationship between score
histories and SGPs and the associated potential for ceiling or floor
effects in the models/model estimates may be necessary. These plots are
meant to serve more as a ``canary in the coal mine'' than as a detailed
indicator.

The bottom left panel shows the empirical distribution of SGPs given
prior scale score deciles in the form of a 10 by 10 cell grid.
Percentages of student growth percentiles between the 10\(^{th}\),
20\(^{th}\), 30\(^{th}\), 40\(^{th}\), 50\(^{th}\), 60\(^{th}\),
70\(^{th}\), 80\(^{th}\), and 90\(^{th}\) percentiles were calculated
based upon the empirical decile of the cohort's prior year scaled score
distribution\footnote{The total students in each for the analyses varies
  depending on grade and subject.}. Deviations from perfect fit are
indicated by red and blue shading. The further above 10 the darker the
red, and the further below 10 the darker the blue. A more detailed
discussion about the reasons for and implications of model misfit for
the various SGP analysis types can be found in the ``Goodness of Fit''
section of the 2015 Utah Student Growth Model report.

The bottom right panel of each plot is a Q-Q plot which compares the
observed distribution of SGPs with the theoretical (uniform)
distribution. An ideal plot here will show black step function lines
that do not deviate greatly from the ideal, red line which traces the 45
degree angle of perfect fit.

As an example, Figure 1 shows the results for 8\(^{th}\) grade ELA as an
example of good model fit. Figure 2 is the fit plot for Secondary Math
III, and demonstrates minor model misfit.

\begin{figure}[htbp]
\centering
\includegraphics{../img/Goodness_of_Fit/ELA.2015/gofSGP_Grade_8.png}
\caption{Goodness of Fit Plot for 2015 8\(^{th}\) Grade ELA: Example of
good model fit.}
\end{figure}

The results in all subjects are excellent with few exceptions.
Deviations from perfect fit are indicated by red and blue shading. The
further \emph{above} 10 the darker the red, and the further \emph{below}
10 the darker the blue. In instances where large deviations from 10
occur, the likely cause is that there is a mass point associated with
certain scale scores that makes it impossible to ``split'' the score at
a dividing point forcing a majority of the scores into an adjacent cell.
This occurs more often in lower grades where fewer prior scores are
available (particularly in the lowest grade when only a single prior is
available). This is the case with all large deviations observed in the
Utah data.

\begin{figure}[htbp]
\centering
\includegraphics{../img/Goodness_of_Fit/SEC_MATH_III.2015/2015_SEC_MATH_III_EOCT;2014_SEC_MATH_II_EOCT;2012_ALGEBRA_I_EOCT.png}
\caption{Goodness of Fit Plot for 2015 Secondary Math III: Example of
slight model mis-fit.}
\end{figure}

\pagebreak

\subsection{Student Level Results}\label{student-level-results}

To investigate the possibility that individual level misfit might impact
summary level results, student growth percentile analyses were run on
all students and the results were examined relative to prior
achievement. With perfect fit to data, the correlation between students'
most recent prior achievement scores and their student growth
percentiles is zero (i.e., the goodness of fit tables would have a
uniform distribution of percentiles across all previous scale score
levels). To investigate in another way, correlations between prior
student scale scores and student growth percentiles were
calculated.\footnote{In addition to providing information about model
  fit, these student level correlations can assess potential impact of
  test ceiling effects.} Additionally, the correlation between prior and
current scale scores (achievement) are calculated to provide a contrast.
For evidence of good model fit, the desired observed relationships would
be no relationship (zero correlation) between prior achievement and
growth, and a positive relationship between prior and current
achievement, which suggests that growth is in fact detectable and growth
modeling is therefore reasonable to begin with.

Student level correlations between the SGP and prior achievement are
presented here. The results are generally as expected. With
cohort-referenced percentiles, when the model is perfectly fit to the
data, the correlation between students' most recent prior achievement
scores and their student growth percentiles is zero (i.e., there is a
uniform distribution of percentiles across all previous scale score
levels). Correlations for Utah cohort-referenced SGPs are all
essentially zero. This provides assurance that the models have fit the
data well, and indicate that students can demonstrate high (or low)
growth regardless of prior achievement using cohort-referenced SGPs.
Furthermore, a strong relationship exists between prior and current
scale scores.

\begin{table}[H]
\caption*{\textbf{Table 5:} EOGT Student Level Correlations between Prior Standardized Scale Score and 1) SGP or 2) Current Scale Score.\label{table5}} 
\begin{center}
\begin{tabular}{lllll}
\hline\hline
\multicolumn{1}{c}{Content Area}&\multicolumn{1}{c}{Grade}&\multicolumn{1}{c}{$\
r_ {  SGP}$}&\multicolumn{1}{c}{$\
r_ { Scale Score}$}&\multicolumn{1}{c}{N Size}\tabularnewline
\hline
ELA& 4&0.00&0.82&44,158\tabularnewline
& 5&0.00&0.83&44,245\tabularnewline
& 6&0.00&0.83&43,635\tabularnewline
& 7&0.00&0.83&41,821\tabularnewline
& 8&0.00&0.85&40,850\tabularnewline
& 9&0.01&0.84&39,097\tabularnewline
&10&0.01&0.83&37,430\tabularnewline
&11&0.00&0.83&34,210\tabularnewline
Mathematics& 4&0.00&0.84&44,265\tabularnewline
& 5&0.00&0.85&44,370\tabularnewline
& 6&0.00&0.84&43,524\tabularnewline
& 7&0.00&0.83&40,604\tabularnewline
& 8&0.00&0.83&39,300\tabularnewline
Science& 5&0.00&0.78&44,421\tabularnewline
& 6&0.00&0.77&43,786\tabularnewline
& 7&0.00&0.79&41,900\tabularnewline
& 8&0.01&0.82&37,714\tabularnewline
\hline
\end{tabular}\end{center}

\end{table}

\begin{table}[H]
\caption*{\textbf{Table 6:} EOCT Student Level Correlations between Prior Standardized Scale Score and 1) SGP or 2) Current Scale Score.\label{table6}} 
\begin{center}
\begin{tabular}{llll}
\hline\hline
\multicolumn{1}{c}{Content Area}&\multicolumn{1}{c}{$\
r_ {  SGP}$}&\multicolumn{1}{c}{$\
r_ { Scale Score}$}&\multicolumn{1}{c}{N Size}\tabularnewline
\hline
Biology&0.01&0.75&33,777\tabularnewline
Chemistry&0.00&0.77&17,165\tabularnewline
Earth Science&0.01&0.80&21,954\tabularnewline
Physics&0.01&0.74&13,337\tabularnewline
Sec Math I&0.01&0.83&37,664\tabularnewline
Sec Math II&0.03&0.79&35,079\tabularnewline
Sec Math III&0.03&0.75&25,646\tabularnewline
\hline
\end{tabular}\end{center}

\end{table}

\subsection{Group Level Results}\label{group-level-results}

Unlike when reporting SGPs at the individual level, when aggregating to
the group level (e.g., school) the correlation between aggregate prior
student achievement and aggregate growth is rarely zero. The correlation
between prior student achievement and growth at the school level is a
compelling descriptive statistic because it indicates whether students
attending schools serving higher achieving students grow faster (on
average) than those students attending schools serving lower achieving
students. Results from previous state analyses show a correlation
between prior achievement of students associated with a current school
(quantified as percent at/above proficient) and the median SGP are
typically between 0.1 and 0.3 (although higher numbers have been
observed in some states as well). That is, these results indicate that
on average, students attending schools serving lower achieving students
tend to demonstrate less exemplary growth than those attending schools
serving higher achieving students. Equivalently, based upon ordinary
least squares (OLS) regression assumptions, the prior achievement level
of students attending a school accounts for between 1 and 10 percent of
the variability observed in student growth. There are no definitive
numbers on what this correlation should be, but recent studies on
value-added models show similar results (McCaffrey, Han, \& Lockwood,
2008).

\subsubsection{School Level Results}\label{school-level-results}

To illustrate these relationships visually, the bubble charts in Figures
3, 4 and 5 depict growth as quantified by the median SGP of students at
the school against achievement/status, quantified by percentage of
student at/above proficient at the school\footnote{Percent Prior
  Proficient in this case is determined by the percent of student's that
  scored in the Proficient or Advanced range of all student's that
  received a score. This measure does not reflect student's that did not
  receive a score.}. The charts have been successful in helping to
motivate the discussion of the two qualities: student achievement and
student growth. Though the figures are not detailed enough to indicate
strength of relationship between growth and achievement, they are
suggestive and valuable for discussions with stakeholders who are being
introduced to the growth model for the first time.

\begin{figure}[htbp]
\centering
\includegraphics{../img/Bubble_Plots/2015/State/Style_1/Utah_2015_ELA_State_Bubble_Plot_(Prior_Achievement).png}
\caption{School Level Bubble Plots for Utah: ELA, 2014-2015.}
\end{figure}

\begin{figure}[htbp]
\centering
\includegraphics{../img/Bubble_Plots/2015/State/Style_1/Utah_2015_Mathematics_State_Bubble_Plot_(Prior_Achievement).png}
\caption{School Level Bubble Plots for Utah: Mathematics, 2014-2015.}
\end{figure}

\begin{figure}[htbp]
\centering
\includegraphics{../img/Bubble_Plots/2015/State/Style_1/Utah_2015_Science_State_Bubble_Plot_(Prior_Achievement).png}
\caption{School Level Bubble Plots for Utah: Science, 2014-2015.}
\end{figure}

The relationship between average prior student achievement and median
SGP observed for Utah is relatively strong compared to some other states
for which the Center has done SGP analyses. Table 7 shows correlations
between prior achievement (measured as the mean prior standardized scale
score as well as the percent at/above proficient at the school\footnote{Percent
  Prior Proficient in this case is determined by the percent of
  student's that scored in the Proficient or Advanced range of all
  student's that received a score. This measure does not reflect
  student's that did not receive a score but are included in the
  denominator of percent proficient.}). All results shown here are for
schools with 10 or more students.

\begin{table}[H]
\caption*{\textbf{Table 7:} School Level Correlations between Mean Prior Standardized Scale Score and 1) Aggregate SGPs or 2) Aggregate Current Scale Score - (Combined Subjects)\label{table7}} 
\begin{center}
\begin{tabular}{llll}
\hline\hline
\multicolumn{1}{c}{Year}&\multicolumn{1}{c}{Median SGP}&\multicolumn{1}{c}{Mean SGP}&\multicolumn{1}{c}{Pct Proficient Or Above}\tabularnewline
\hline
2011&0.47&0.47&0.81\tabularnewline
2012&0.45&0.45&0.80\tabularnewline
2013&0.30&0.32&0.86\tabularnewline
2014&0.41&0.41&0.89\tabularnewline
2015&0.34&0.36&0.93\tabularnewline
\hline
\end{tabular}\end{center}

\end{table}

Correlation tables describing the relationship between prior achievement
(again defined as mean prior standardized scale score) and aggregate
growth percentiles are presented below in separate subsections for grade
level and EOCT subjects. Additionally, the correlation between the
growups prior achievement and a measure of their current achievement
(here the percent of kids that are at or above the proficiency cuts).
Typically these correlations are stronger than that between prior
achievement and growth, which suggests that school achievement
\emph{status} tends to stay the same over time.

This is indeed what we see in the correlation tables. The first table in
the each subsection provides these overall SGP aggregates' relationships
with mean prior standardized scale scores. The additional correlation
tables are dis-aggregated by content area, and content area and grade to
provide more detail.

\begin{table}[H]
\caption*{\textbf{Table 8:} School Level EOGT Correlations between Mean Prior Standardized Scale Score and 1) Aggregate SGPs or 2) Aggregate Current Scale Score by Content Area.\label{table8}} 
\begin{center}
\begin{tabular}{lllll}
\hline\hline
\multicolumn{1}{c}{Content Area}&\multicolumn{1}{c}{Year}&\multicolumn{1}{c}{Median SGP}&\multicolumn{1}{c}{Mean SGP}&\multicolumn{1}{c}{Pct Proficient Or Above}\tabularnewline
\hline
ELA&2013&0.36&0.40&0.82\tabularnewline
&2014&0.40&0.42&0.87\tabularnewline
&2015&0.31&0.32&0.93\tabularnewline
Mathematics&2013&0.27&0.28&0.84\tabularnewline
&2014&0.24&0.24&0.88\tabularnewline
&2015&0.24&0.25&0.89\tabularnewline
Science&2013&0.21&0.21&0.90\tabularnewline
&2014&0.05&0.07&0.88\tabularnewline
&2015&0.11&0.12&0.89\tabularnewline
\hline
\end{tabular}\end{center}

\end{table}

\begin{table}[H]
\caption*{\textbf{Table 9:} 2013 to 2015 School Level EOCT Correlations between Mean Prior Standardized Scale Score and 1) Aggregate SGPs or 2) Aggregate Current Scale Score by Content Area.\label{table9}} 
\begin{center}
\begin{tabular}{lllll}
\hline\hline
\multicolumn{1}{c}{Content Area}&\multicolumn{1}{c}{Year}&\multicolumn{1}{c}{Median SGP}&\multicolumn{1}{c}{Mean SGP}&\multicolumn{1}{c}{Pct Proficient Or Above}\tabularnewline
\hline
Earth Science&2013& 0.11& 0.13&0.86\tabularnewline
&2014&-0.13&-0.13&0.80\tabularnewline
&2015&-0.10&-0.10&0.80\tabularnewline
Biology&2013& 0.26& 0.26&0.80\tabularnewline
&2014& 0.13& 0.14&0.67\tabularnewline
&2015& 0.14& 0.15&0.69\tabularnewline
Chemistry&2013& 0.15& 0.17&0.70\tabularnewline
&2014& 0.08& 0.09&0.56\tabularnewline
&2015& 0.23& 0.25&0.82\tabularnewline
Physics&2013& 0.42& 0.40&0.83\tabularnewline
&2014& 0.28& 0.27&0.78\tabularnewline
&2015& 0.22& 0.21&0.75\tabularnewline
Sec Math I&2014& 0.22& 0.25&0.75\tabularnewline
&2015& 0.17& 0.15&0.83\tabularnewline
Sec Math II&2014& 0.47& 0.48&0.76\tabularnewline
&2015& 0.41& 0.45&0.90\tabularnewline
Sec Math III&2014& 0.20& 0.16&0.71\tabularnewline
&2015& 0.32& 0.36&0.90\tabularnewline
\hline
\end{tabular}\end{center}

\end{table}

The final table disaggregates the 2015 correlations for the EOGT
subjects further by grade level.

\begin{table}[H]
\caption*{\textbf{Table 10:} 2015 School Level EOGT Correlations between Mean Prior Standardized Scale Score and 1) Aggregate SGPs or 2) Aggregate Current Scale Score by Grade.\label{table10}} 
\begin{center}
\begin{tabular}{lllll}
\hline\hline
\multicolumn{1}{c}{Content Area}&\multicolumn{1}{c}{Grade}&\multicolumn{1}{c}{Median SGP}&\multicolumn{1}{c}{Mean SGP}&\multicolumn{1}{c}{Pct Proficient Or Above}\tabularnewline
\hline
ELA& 4& 0.04& 0.05&0.83\tabularnewline
& 5& 0.17& 0.16&0.86\tabularnewline
& 6& 0.13& 0.14&0.86\tabularnewline
& 7& 0.17& 0.18&0.86\tabularnewline
& 8& 0.13& 0.14&0.86\tabularnewline
& 9& 0.25& 0.24&0.85\tabularnewline
&10& 0.18& 0.13&0.88\tabularnewline
&11& 0.20& 0.21&0.87\tabularnewline
Mathematics& 4& 0.01& 0.03&0.79\tabularnewline
& 5& 0.10& 0.11&0.83\tabularnewline
& 6& 0.02& 0.02&0.79\tabularnewline
& 7& 0.20& 0.20&0.85\tabularnewline
& 8& 0.14& 0.14&0.81\tabularnewline
Science& 5& 0.06& 0.06&0.82\tabularnewline
& 6&-0.02&-0.02&0.75\tabularnewline
& 7& 0.12& 0.13&0.74\tabularnewline
& 8&-0.03&-0.03&0.84\tabularnewline
\hline
\end{tabular}\end{center}

\end{table}

\pagebreak

\section*{References}\label{references}
\addcontentsline{toc}{section}{References}

Betebenner, D. W. (2009). Norm- and criterion-referenced student growth.
\emph{Educational Measurement: Issues and Practice}, \emph{28}(4),
42--51.

Betebenner, D. W., {VanIwaarden}, A., Domingue, B., \& Shang, Y. (2015).
\emph{SGP: An r package for the calculation and visualization of student
growth percentiles \& percentile growth trajectories.} Retrieved from
\url{https://github.com/CenterForAssessment/SGP}

Chernozhukov, V., Fern{á}ndez-Val, I., \& Galichon, A. (2010). Quantile
and probability curves without crossing. \emph{Econometrica},
\emph{78}(3), 1093--1125. Wiley Online Library.

Koenker, R. (2005). \emph{Quantile regression}. Cambridge: Cambridge
University Press.

McCaffrey, D., Han, B., \& Lockwood, J. (2008). From data to bonuses: A
case study of the issues related to awarding teachers pay on the basis
of their students' progress. In \emph{National center for performance
incentives conference performance incentives: Their growing impact on
american k-12 education, vanderbilt university, nashville, tN}.

R Development Core Team. (2015). \emph{R: A language and environment for
statistical computing}. Vienna, Austria: R Foundation for Statistical
Computing. Retrieved from \url{http://www.R-project.org}

\bibliographystyle{plainnat}
\bibliography{/Library/Frameworks/R.framework/Versions/3.2/Resources/library/SGPreports/rmarkdown/templates/multi_document/resources/educ.bib}




\end{document}


