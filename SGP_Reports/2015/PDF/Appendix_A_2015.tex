\documentclass[12pt]{article}
\usepackage[margin=0.915in]{geometry}

%%%
%%%             BEGIN Damian Preamble
%%%   Preface commands for pdfLaTeX output
%%%

%%
%% Calling relevant packages
%%

\usepackage[%
pdfstartview={FitV}, 
colorlinks=true,
menucolor=DarkGray,
linkcolor=MidnightBlue,
citecolor=MidnightBlue,
urlcolor=OrangeRed]{hyperref}
\usepackage{graphicx}
\DeclareGraphicsExtensions{.pdf}

%% From pandoc default: Begin
\usepackage{graphicx}
\makeatletter
\def\maxwidth{\ifdim\Gin@nat@width>\linewidth\linewidth\else\Gin@nat@width\fi}
\def\maxheight{\ifdim\Gin@nat@height>\textheight\textheight\else\Gin@nat@height\fi}
\makeatother
% Scale images if necessary, so that they will not overflow the page
% margins by default, and it is still possible to overwrite the defaults
% using explicit options in \includegraphics[width, height, ...]{}
\setkeys{Gin}{width=\maxwidth,height=\maxheight,keepaspectratio}
 %% From pandoc default: Begin

%\usepackage[hyper]{apacite}
\usepackage[svgnames]{xcolor}
\usepackage{rotating,bm,amsmath,amsfonts,amssymb,indentfirst,lscape,fancybox,fancyvrb,listings,pdfpages}
\usepackage[pagestyles]{titlesec}
%\usepackage{ucs}

%%
%% Format changes for chapter and section commands
%%

%\input{onesidedheader.tex}

%%
%% Chapter declarations
%%


%%
%% Section declarations
%%

%%
%% Subsection declarations
%%


%%
%% Subsubsection declarations
%%

%%
%% Listings setup information
%%

\lstset{language=R, frame=ltrb, framesep=5pt, xleftmargin=12pt, xrightmargin=5pt,
       numbers=none, breaklines=true, fancyvrb=true,
       breakatwhitespace=true, captionpos=b, abovecaptionskip=1.5ex,
       backgroundcolor=\color{Cornsilk},
       basicstyle=\small\color{DarkSlateGrey},
       keywordstyle=\ttfamily\color{DarkSlateGrey},
       identifierstyle=\ttfamily\color{DarkSlateGrey}\bfseries, 
       commentstyle=\color{Fuchsia},
       stringstyle=\ttfamily,
       showstringspaces=false}

%%
%% Header and Footer Specification 
%% TO ACTIVATE THE HEADER/FOOTER, ONE MUST PLACE \pagestyle{plain} 
%% followed by \pagestyle{damian} in the document
%%

\widenhead{0.14in}{0.14in}

\renewpagestyle{plain}{}

\newpagestyle{damian}[\sffamily]{
\headrule
\sethead[\sectiontitle][\chaptertitle][\thepage]
  {\sectiontitle}{\chaptertitle}{\thepage}

\footrule
\setfoot[][\raisebox{-.85ex}[0pt]{\NavigationBar}][]
{}{\raisebox{-.85ex}[0pt]{\NavigationBar}}{}
	\newcommand{\NavigationBar}{%
	  \Acrobatmenu{PrevPage}{Previous}\hspace{.5cm}
	  \Acrobatmenu{NextPage}{Next}\hspace{.5cm}
	  \Acrobatmenu{FirstPage}{First}\hspace{.5cm}
	  \Acrobatmenu{LastPage}{Last}\hspace{.5cm}
	  \Acrobatmenu{GoBack}{Back}\hspace{.5cm}
	  \Acrobatmenu{Quit}{Quit}%
}
}

%%
%% Custom specifications, commands, and colors
%%

\definecolor{DarkGray}{cmyk}{0,0,0,.624}
\newcommand{\R}{{\sffamily \textup{R}}}
\newcommand{\MlwiN}{{\sffamily \textup{MlwiN}}}
\newcommand{\Sweave}{{\sffamily \textup{Sweave}}}
\newcommand{\sssty}{\scriptscriptstyle}
\newcommand{\bigdot}{\ensuremath{\hspace{-.3ex}\bm{.}\hspace{-.05ex}}}
\renewcommand{\abstractname}{Abstract}
\renewcommand{\lstlistingname}{\R~Code Example}
\renewcommand{\arraystretch}{1.2}
\setlength{\fboxsep}{4mm}
\setlength{\fboxrule}{0.4pt}

%%
%%		END Damian Preamble
%%

\hypersetup{%
  pdftitle={ Appendix A to the 2015 Utah Growth Model Report },
  pdfauthor={},
  pdfcreator={},
  pdfkeywords={},
  bookmarks=true} % pdfproducer={pdfLaTeX}

\usepackage{caption}
\usepackage{float}
\usepackage{longtable}
\usepackage{booktabs}
\usepackage{subcaption}
\usepackage{dcolumn}

\setcounter{secnumdepth}{3}




%\usepackage{pdfdraftcopy}
\newtheorem{proposition}{Proposition}
\newtheorem{theorem}{Theorem}
\newtheorem{definition}{Definition}
\newtheorem{corollary}{Corollary}
\DeclareMathOperator*{\argmin}{arg\,min}

% \usepackage{bbm}
\DeclareMathAlphabet{\mathbbm}{U}{bbm}{m}{n}
\SetMathAlphabet\mathbbm{bold}{U}{bbm}{bx}{n}
\DeclareMathAlphabet{\mathbbmss}{U}{bbmss}{m}{n}
\SetMathAlphabet\mathbbmss{bold}{U}{bbmss}{bx}{n}
\DeclareMathAlphabet{\mathbbmtt}{U}{bbmtt}{m}{n}


\newcommand{\pl}[1]{\textsf{PL#1}}
\newcommand{\Cov}{\ensuremath{\mbox{\textsf{Cov}}}}
\newcommand{\Diag}{\ensuremath{\mbox{\textsf{Diag}}}}
\newcommand{\Bias}{\ensuremath{\mbox{\textsf{Bias}}}}
\newcommand{\Astar}[1]{\ensuremath{#1^{^*}}}
\thispagestyle{plain}
\pagestyle{damian}

\begin{document}

\title{\textsf{\LARGE Appendix A to the 2015 Utah Growth Model Report  \\\medskip Utah Student Growth Model Fit Plots }}
\author{}

 \date{October 2015} 

\maketitle

\newpage


\section{Student Growth Percentile Fit
Plots}\label{student-growth-percentile-fit-plots}

A goodness of fit plot is produced for each unique analysis run in 2015.
Each analysis is defined by the grade and content area for the
grade-level analyses and the unique course progression sequences for the
end of course test (EOCT) subjects.

Most fit plot contains four panels. When the prior scale score is
unavailable the top panel will be excluded. Usually unavailability is
due to the use of equated SGP estimation in an assessment program
transition year or an End of Course Test (EOCT) analyses that use a
prior course progression that is not a subset of the most typical (i.e.
``canonical'') course progression. Prior scale score and prior
proficiency data is required in the top panel as it displays a mosaic
plot that shows the percentage of students that fall into each
proficiency level, and the location of the 10\(^{th}\) through
90\(^{th}\) quantiles of the Student Growth Percentile (SGP)
distribution represented as dashed white lines (with the exception of
the solid white line for the median/50\(^{th}\) percentile). Ideally
this plot will show that the median percentile is at or near 50 for all
prior achievement level groups.

The middle panel contains a ``Ceiling/Floor Effects Test'', which is a
relatively recent addition to the goodness of fit plots. It is intended
to help identify potential problems in SGP estimation at the Highest and
Lowest Obtainable (or Observed) Scale Scores (HOSS and LOSS). Issues can
occur here where, when ceiling or floor effects are present in both the
current and prior year(s) scores, it may be relatively typical for
extremely high/low achieving students to consistently score at or near
the HOSS/LOSS. As a result, the SGPs for students scoring at the
HOSS/LOSS will be unexpectedly low/high. That is, for example, if a
sufficient number of students maintain performance at the HOSS over
time, this performance will be estimated to typical, and therfore SGP
estimates will reflect typical growth (e.g.~50th percentile). In some
cases these extreme score values or small deviations from them might
even yield low growth estimates. Although these score patterns can
ligitimately be estimated as a low growth percentiles because they
represent rather typical growth, it is potentially an unfair description
of their growth performance (and by proxy teacher, school or district
performance or ``value added'') caused by an artifact of the inability
of the assessment to adequately measure student performance at extreme
ability levels.

The table of values here shows whether the current year scale scores at
both extremes yield the expected SGPs\footnote{Note that the prior year
  scale scores are not represented here, but are critical in the SGP
  calculation of all students}. The expectation is that the majority of
SGPs for students scoring at or near the LOSS will be low (preferably
less than 5 and not higher than 10), and that SGPs for students scoring
at or near the HOSS will be high (preferably higher than 95 and not less
than 90). Because few students may score \emph{exactly} at the
HOSS/LOSS\footnote{This is particularly true when IRT Theta (\(\theta\))
  estimates are used rather than scaled scores, which often apply
  artificial LOSS/HOSS cut points.}, the top/bottom 50 students are
selected and any student scoring within their range of scores are
selected for inclusion in these tables. Consequently, there may be a
range of scores at the HOSS/LOSS rather than a single score\footnote{This
  can make the interpretation of the SGP distribution somewhat harder
  because score not directly at the extremes do not necessarily preclude
  maximum SGP estimates.}, and there may be more than 50 students
included in the HOSS/LOSS row if the 50 students at the extremes only
contain the single HOSS/LOSS score\footnote{This also leads to potential
  difficulties in interpretation because with a higher number of
  students comes a greater distribution of prior scale scores and
  therefore a greater distribution of the expected SGPs}. In either
case, a more fine grained analysis of the relationship between score
histories and SGPs and the associated potential for ceiling or floor
effects in the models/model estimates may be necessary. These plots are
meant to serve more as a ``canary in the coal mine'' than as a detailed
indicator.

The bottom left panel shows the empirical distribution of SGPs given
prior scale score deciles in the form of a 10 by 10 cell grid.
Percentages of student growth percentiles between the 10\(^{th}\),
20\(^{th}\), 30\(^{th}\), 40\(^{th}\), 50\(^{th}\), 60\(^{th}\),
70\(^{th}\), 80\(^{th}\), and 90\(^{th}\) percentiles were calculated
based upon the empirical decile of the cohort's prior year scaled score
distribution\footnote{The total students in each for the analyses varies
  depending on grade and subject.}. Deviations from perfect fit are
indicated by red and blue shading. The further above 10 the darker the
red, and the further below 10 the darker the blue. A more detailed
discussion about the reasons for and implications of model misfit for
the various SGP analysis types can be found in the ``Goodness of Fit''
section of the 2015 Utah Student Growth Model report.

The bottom right panel of each plot is a Q-Q plot which compares the
observed distribution of SGPs with the theoretical (uniform)
distribution. An ideal plot here will show black step function lines
that do not deviate greatly from the ideal, red line which traces the 45
degree angle of perfect fit.

\section{SAGE Grade-Level Fit Plots}\label{sage-grade-level-fit-plots}

\subsection{ELA}\label{ela}

\begin{figure}[htbp]
\centering
\includegraphics{../img/Goodness_of_Fit/ELA.2015/2015_ELA_4;2014_ELA_3.png}
\caption{Goodness of Fit Plot for Grade 4 ELA, 2015.}
\end{figure}

\begin{figure}[htbp]
\centering
\includegraphics{../img/Goodness_of_Fit/ELA.2015/2015_ELA_5;2014_ELA_4;2013_ELA_3.png}
\caption{Goodness of Fit Plot for Grade 5 ELA, 2015.}
\end{figure}

\begin{figure}[htbp]
\centering
\includegraphics{../img/Goodness_of_Fit/ELA.2015/2015_ELA_6;2014_ELA_5;2013_ELA_4;2012_ELA_3.png}
\caption{Goodness of Fit Plot for Grade 4 ELA, 2015.}
\end{figure}

\begin{figure}[htbp]
\centering
\includegraphics{../img/Goodness_of_Fit/ELA.2015/2015_ELA_7;2014_ELA_6;2013_ELA_5;2012_ELA_4;2011_ELA_3.png}
\caption{Goodness of Fit Plot for Grade 7 ELA, 2015.}
\end{figure}

\begin{figure}[htbp]
\centering
\includegraphics{../img/Goodness_of_Fit/ELA.2015/2015_ELA_8;2014_ELA_7;2013_ELA_6;2012_ELA_5;2011_ELA_4;2010_ELA_3.png}
\caption{Goodness of Fit Plot for Grade 8 ELA, 2015.}
\end{figure}

\begin{figure}[htbp]
\centering
\includegraphics{../img/Goodness_of_Fit/ELA.2015/2015_ELA_9;2014_ELA_8;2013_ELA_7;2012_ELA_6;2011_ELA_5;2010_ELA_4.png}
\caption{Goodness of Fit Plot for Grade 9 ELA, 2015.}
\end{figure}

\begin{figure}[htbp]
\centering
\includegraphics{../img/Goodness_of_Fit/ELA.2015/2015_ELA_10;2014_ELA_9;2013_ELA_8;2012_ELA_7;2011_ELA_6;2010_ELA_5.png}
\caption{Goodness of Fit Plot for Grade 10 ELA, 2015.}
\end{figure}

\begin{figure}[htbp]
\centering
\includegraphics{../img/Goodness_of_Fit/ELA.2015/2015_ELA_11;2014_ELA_10;2013_ELA_9;2012_ELA_8;2011_ELA_7;2010_ELA_6.png}
\caption{Goodness of Fit Plot for Grade 11 ELA, 2015.}
\end{figure}

\clearpage 

\subsection{Mathematics}\label{mathematics}

\begin{figure}[htbp]
\centering
\includegraphics{../img/Goodness_of_Fit/MATHEMATICS.2015/2015_MATH_4;2014_MATH_3.png}
\caption{Goodness of Fit Plot for Grade 4 Mathematics, 2015.}
\end{figure}

\begin{figure}[htbp]
\centering
\includegraphics{../img/Goodness_of_Fit/MATHEMATICS.2015/2015_MATH_5;2014_MATH_4;2013_MATH_3.png}
\caption{Goodness of Fit Plot for Grade 5 Mathematics, 2015.}
\end{figure}

\begin{figure}[htbp]
\centering
\includegraphics{../img/Goodness_of_Fit/MATHEMATICS.2015/2015_MATH_6;2014_MATH_5;2013_MATH_4;2012_MATH_3.png}
\caption{Goodness of Fit Plot for Grade 6 Mathematics, 2015.}
\end{figure}

\begin{figure}[htbp]
\centering
\includegraphics{../img/Goodness_of_Fit/MATHEMATICS.2015/2015_MATH_7;2014_MATH_6;2013_MATH_5;2012_MATH_4;2011_MATH_3.png}
\caption{Goodness of Fit Plot for Grade 7 Mathematics, 2015.}
\end{figure}

\begin{figure}[htbp]
\centering
\includegraphics{../img/Goodness_of_Fit/MATHEMATICS.2015/2015_MATH_8;2014_MATH_7;2013_MATH_6;2012_MATH_5;2011_MATH_4;2010_MATH_3.png}
\caption{Goodness of Fit Plot for Grade 8 Mathematics, 2015.}
\end{figure}

\clearpage 

\subsection{Science}\label{science}

\begin{figure}[htbp]
\centering
\includegraphics{../img/Goodness_of_Fit/SCIENCE.2015/2015_SCIENCE_5;2014_SCIENCE_4.png}
\caption{Goodness of Fit Plot for Grade 5 Science, 2015.}
\end{figure}

\begin{figure}[htbp]
\centering
\includegraphics{../img/Goodness_of_Fit/SCIENCE.2015/2015_SCIENCE_6;2014_SCIENCE_5;2013_SCIENCE_4.png}
\caption{Goodness of Fit Plot for Grade 6 Science, 2015.}
\end{figure}

\begin{figure}[htbp]
\centering
\includegraphics{../img/Goodness_of_Fit/SCIENCE.2015/2015_SCIENCE_7;2014_SCIENCE_6;2013_SCIENCE_5;2012_SCIENCE_4.png}
\caption{Goodness of Fit Plot for Grade 7 Science, 2015.}
\end{figure}

\begin{figure}[htbp]
\centering
\includegraphics{../img/Goodness_of_Fit/SCIENCE.2015/2015_SCIENCE_8;2014_SCIENCE_7;2013_SCIENCE_6;2012_SCIENCE_5;2011_SCIENCE_4.png}
\caption{Goodness of Fit Plot for Grade 8 Science, 2015.}
\end{figure}

\pagebreak

\section{SAGE EOCT Fit Plots}\label{sage-eoct-fit-plots}

\subsection{Earth Science}\label{earth-science}

\begin{figure}[htbp]
\centering
\includegraphics{../img/Goodness_of_Fit/EARTH_SCIENCE.2015/2015_EARTH_SCIENCE_EOCT;2014_SCIENCE_8;2013_SCIENCE_7;2012_SCIENCE_6;2011_SCIENCE_5;2010_SCIENCE_4.png}
\caption{Goodness of Fit Plot for 2015 Earth Science (Priors - 2014
Science Grade 8, 2013 Science Grade 7, 2012 Science Grade 6, 2011
Science Grade 5, 2010 Science Grade 4)}
\end{figure}

\clearpage 

\subsection{Biology}\label{biology}

\subsubsection{Earth Science as Imediate
Prior}\label{earth-science-as-imediate-prior}

\begin{figure}[htbp]
\centering
\includegraphics{../img/Goodness_of_Fit/BIOLOGY.2015/2015_BIOLOGY_EOCT;2014_EARTH_SCIENCE_EOCT;2013_SCIENCE_8;2012_SCIENCE_7;2011_SCIENCE_6;2010_SCIENCE_5.png}
\caption{Goodness of Fit Plot for 2015 Biology (Priors - 2014 Earth
Science, 2013 Science Grade 8, 2012 Science Grade 7, 2011 Science Grade
6, 2010 Science Grade 5)}
\end{figure}

\clearpage 

\subsubsection{Grade 8 Science as Imediate
Prior}\label{grade-8-science-as-imediate-prior}

\begin{figure}[htbp]
\centering
\includegraphics{../img/Goodness_of_Fit/BIOLOGY.2015/2015_BIOLOGY_EOCT;2014_SCIENCE_8;2013_SCIENCE_7;2012_SCIENCE_6;2011_SCIENCE_5;2010_SCIENCE_4.png}
\caption{Goodness of Fit Plot for 2015 Biology (Priors - 2014 Science
Grade 8, 2013 Science Grade 7, 2012 Science Grade 6, 2011 Science Grade
5, 2010 Science Grade 4)}
\end{figure}

\clearpage 

\subsection{Chemistry}\label{chemistry}

\subsubsection{Biology as Imediate Prior, Grade 8 Science
Secondary}\label{biology-as-imediate-prior-grade-8-science-secondary}

\begin{figure}[htbp]
\centering
\includegraphics{../img/Goodness_of_Fit/CHEMISTRY.2015/2015_CHEMISTRY_EOCT;2014_BIOLOGY_EOCT;2013_SCIENCE_8;2012_SCIENCE_7;2011_SCIENCE_6;2010_SCIENCE_5.png}
\caption{Goodness of Fit Plot for 2015 Chemistry (Priors - 2014 Biology,
2013 Science Grade 8, 2012 Science Grade 7, 2011 Science Grade 6, 2010
Science Grade 5)}
\end{figure}

\clearpage 

\subsubsection{Biology as Imediate Prior, Earth Science
Secondary}\label{biology-as-imediate-prior-earth-science-secondary}

\begin{figure}[htbp]
\centering
\includegraphics{../img/Goodness_of_Fit/CHEMISTRY.2015/2015_CHEMISTRY_EOCT;2014_BIOLOGY_EOCT;2013_EARTH_SCIENCE_EOCT;2012_SCIENCE_8;2011_SCIENCE_7;2010_SCIENCE_6.png}
\caption{Goodness of Fit Plot for 2015 Chemistry (Priors - 2014 Biology,
2013 Earth Science, 2012 Science Grade 8, 2011 Science Grade 7, 2010
Science Grade 6)}
\end{figure}

\clearpage 

\subsection{Physics}\label{physics}

\subsubsection{Biology as Imediate
Prior}\label{biology-as-imediate-prior}

\begin{figure}[htbp]
\centering
\includegraphics{../img/Goodness_of_Fit/PHYSICS.2015/2015_PHYSICS_EOCT;2014_BIOLOGY_EOCT;2013_EARTH_SCIENCE_EOCT;2012_SCIENCE_EOCT;2011_SCIENCE_EOCT;2010_SCIENCE_EOCT.png}
\caption{Goodness of Fit Plot for 2015 Physics (Priors - 2014 Biology,
2013 Earth Science, 2012 Science, 2011 Science, 2010 Science)}
\end{figure}

\begin{figure}[htbp]
\centering
\includegraphics{../img/Goodness_of_Fit/PHYSICS.2015/2015_PHYSICS_EOCT;2014_BIOLOGY_EOCT;2013_SCIENCE_EOCT;2012_SCIENCE_EOCT;2011_SCIENCE_EOCT;2010_SCIENCE_EOCT.png}
\caption{Goodness of Fit Plot for 2015 Physics (Priors - 2014 Biology,
2013 Science, 2012 Science, 2011 Science, 2010 Science)}
\end{figure}

\clearpage 

\subsubsection{Chemistry as Imediate
Prior}\label{chemistry-as-imediate-prior}

\begin{figure}[htbp]
\centering
\includegraphics{../img/Goodness_of_Fit/PHYSICS.2015/2015_PHYSICS_EOCT;2014_CHEMISTRY_EOCT;2013_BIOLOGY_EOCT;2012_EARTH_SCIENCE_EOCT;2011_SCIENCE_EOCT;2010_SCIENCE_EOCT.png}
\caption{Goodness of Fit Plot for 2015 Physics (Priors - 2014 Chemistry,
2013 Biology, 2012 Earth Science, 2011 Science, 2010 Science)}
\end{figure}

\begin{figure}[htbp]
\centering
\includegraphics{../img/Goodness_of_Fit/PHYSICS.2015/2015_PHYSICS_EOCT;2014_CHEMISTRY_EOCT;2013_BIOLOGY_EOCT;2012_SCIENCE_8;2011_SCIENCE_7;2010_SCIENCE_6.png}
\caption{Goodness of Fit Plot for 2015 Physics (Priors - 2014 Chemistry,
2013 Biology, 2012 Science Grade 8, 2011 Science 7, 2010 Science 6)}
\end{figure}

\clearpage 

\subsection{Secondary Math I}\label{secondary-math-i}

\begin{figure}[htbp]
\centering
\includegraphics{../img/Goodness_of_Fit/SEC_MATH_I.2015/2015_SEC_MATH_I_EOCT;2014_MATH_8;2013_MATH_7;2012_MATH_6;2011_MATH_5;2010_MATH_4.png}
\caption{Goodness of Fit Plot for 2015 Secondary Math I (Priors - 2014
Math 8, 2013 Math Grade 7, 2012 Math Grade 6, 2011 Math Grade 5, 2010
Math Grade 4)}
\end{figure}

\clearpage 

\subsection{Secondary Math II}\label{secondary-math-ii}

\begin{figure}[htbp]
\centering
\includegraphics{../img/Goodness_of_Fit/SEC_MATH_II.2015/2015_SEC_MATH_II_EOCT;2014_SEC_MATH_I_EOCT;2013_PRE_ALGEBRA_EOCT;2012_MATH_7;2011_MATH_6;2010_MATH_5.png}
\caption{Goodness of Fit Plot for 2015 Secondary Math II (Priors - 2014
Sec Math I, 2013 Pre Algebra, 2012 Math Grade 7, 2011 Math Grade 6, 2010
Math Grade 5)}
\end{figure}

\begin{figure}[htbp]
\centering
\includegraphics{../img/Goodness_of_Fit/SEC_MATH_II.2015/2015_SEC_MATH_II_EOCT;2014_SEC_MATH_I_EOCT.png}
\caption{Goodness of Fit Plot for 2015 Secondary Math II (Priors - 2014
Sec Math I)}
\end{figure}

\clearpage 

\subsection{Secondary Math III}\label{secondary-math-iii}

\begin{figure}[htbp]
\centering
\includegraphics{../img/Goodness_of_Fit/SEC_MATH_III.2015/2015_SEC_MATH_III_EOCT;2014_SEC_MATH_II_EOCT;2012_ALGEBRA_I_EOCT.png}
\caption{Goodness of Fit Plot for 2015 Secondary Math III (Priors - 2014
Sec Math II, 2012 Algebra I)}
\end{figure}

\bibliographystyle{plainnat}
\bibliography{/Library/Frameworks/R.framework/Versions/3.2/Resources/library/SGPreports/rmarkdown/templates/multi_document/resources/educ.bib}




\end{document}


